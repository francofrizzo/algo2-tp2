\section{M\'odulo \'Arbol binario}

\Encabezado{Notas preliminares}
  En todos los casos, al indicar las complejidades de los algoritmos, las variables que se utilizan corresponden a:
  \vspace{-0.5em}\begin{itemize}
    \item $n$: Cantidad de nodos en el \'arbol binario.
  \end{itemize}

\begin{Interfaz}
  
  \begin{paramFormales}
    \paramGeneros{$\alpha$}
  \end{paramFormales}

  \seExplicaCon{Arbol binario($\alpha$)}

  \generos{\tipo{ab($\alpha$)}}

  \Encabezado{Operaciones de \'arbol binario}

    \InterfazFuncion{Nil}{}{ab($\alpha$)}
    {$res \igobs$ nil}
    [$\Theta(1)$]
    [Crea y devuelve un \'arbol binario vac\'io.]

    \InterfazFuncion{Bin}{\In{i}{ab($\alpha$)}, \In{r}{$\alpha$}, \In{d}{ab($\alpha$)}}{ab($\alpha$)}
    {$res$ $\igobs$ bin($i$, $r$, $d$) $\land$ esAlias(izq($res$), $i$) $\land$ esAlias(raiz($res$), $r$) $\land$ esAlias(der($res$), $d$)}
    [$\Theta(1)$]
    [Crea y devuelve un \'arbol binario usando los par\'ametros de entrada como sub\'arbol izquierdo, ra\'iz y sub\'arbol derecho, respectivamente.]
    [Tanto la ra\'iz como los dos sub\'arboles son tomados por referencia. Cualquier modificaci\'on de los mismos incide sobre el \'arbol binario creado.]

    \InterfazFuncion{EsNil}{\In{a}{ab($\alpha$)}}{bool}
    {$res$ $\igobs$ nil?($a$)}
    [$\Theta(1)$]
    [Devuelve \emph{true} si y solo si el \'arbol binario est\'a vac\'io.]

    \InterfazFuncion{Raiz}{\In{a}{ab($\alpha$)}}{$\alpha$}
    [$\neg$ nil?($a$)]
    {esAlias($res$, raiz($a$))}
    [$\Theta(1)$]
    [Devuelve la ra\'iz del \'arbol binario pasado por par\'ametro.]
    [El elemento se devuelve por referencia.]

    \InterfazFuncion{Izq}{\In{a}{ab($\alpha$)}}{$\alpha$}
    [$\neg$ nil?($a$)]
    {esAlias($res$, izq($a$))}
    [$\Theta(1)$]
    [Devuelve el sub\'arbol izquierdo del \'arbol binario pasado por par\'ametro.]
    [El sub\'arbol se devuelve por referencia.]

    \InterfazFuncion{Der}{\In{a}{ab($\alpha$)}}{$\alpha$}
    [$\neg$ nil?($a$)]
    {esAlias($res$, der($a$))}
    [$\Theta(1)$]
    [Devuelve el sub\'arbol derecho del \'arbol binario pasado por par\'ametro.]
    [El sub\'arbol se devuelve por referencia.]

    \InterfazFuncion{Altura}{\In{a}{ab($\alpha$)}}{nat}
    {$res \igobs$ altura($a$))}
    [$\Theta(1)$]
    [Devuelve la m\'axima distancia entre la ra\'iz del \'arbol binario y alguna de sus hojas.]

    \InterfazFuncion{CantNodos}{\In{a}{ab($\alpha$)}}{nat}
    {$res \igobs$ tama\~no($a$))}
    [$\Theta(1)$]
    [Devuelve la cantidad de nodos del \'arbol binario.]

    \InterfazFuncion{Inorder}{\In{a}{ab($\alpha$)}}{lista($\alpha$)}
    {$res \igobs$ inorder($a$))}
    [$\Theta(n)$]
    [Devuelve una lista con todos los elementos del \'arbol, recorridos en inorden.]

    \InterfazFuncion{Preorder}{\In{a}{ab($\alpha$)}}{lista($\alpha$)}
    {$res \igobs$ preorder($a$))}
    [$\Theta(n)$]
    [Devuelve una lista con todos los elementos del \'arbol, recorridos en preorden.]

    \InterfazFuncion{Postorder}{\In{a}{ab($\alpha$)}}{lista($\alpha$)}
    {$res \igobs$ postorder($a$))}
    [$\Theta(n)$]
    [Devuelve una lista con todos los elementos del \'arbol, recorridos en postorden.]

\end{Interfaz}

\begin{Representacion}

  \begin{Estructura}{ab($\alpha$)}[puntero(nodo)]

    \begin{Tupla}[nodo]
      \tupItemNL{valor}{$\alpha$}%
      % \tupItemNL{padre}{puntero(nodo)}%
      \tupItemNL{izq}{puntero(nodo)}%
      \tupItemNL{der}{puntero(nodo)}%
      % \tupItemNL{cantNodosSubarbol}{nat}%
      % \tupItemNL{altSubarbol}{nat}%
    \end{Tupla}

  \end{Estructura} 

  \Rep[puntero(nodo)][a]{
    $\emptyset$?(padres($a$, nodos($a$))) $\land$ \\
    $(\paratodo{nodo}{n})$ ($n$ $\in$ nodos($a$) $\implies$ (\\
    ($(\&n \neq a)$ $\implies$ \#(padres($n$, nodos($a$))) = 1) $\land$ \\
    $n.izq$ $\neq$ NULL $\implies$ $n.izq$ $\neq$ $n.der$ $\land$ \\
    % $n.cantNodosSubarbol$ = \#(nodos($\&n$)) $\land$ \\
    % $n.altSubarbol$ = altura($\&n$) ))
  }
  ~
  \AbsFc[puntero(nodo)]{ab($\alpha$)}[a]{
    \IF $a$ = NULL THEN nil ELSE bin(Abs($a \to izq$), $a \to valor$, Abs($a \to der$)) FI
  }
  ~
  \tadOperacion{hijos}{nodo}{conj(nodo)}{}
  \tadAxioma{hijos($n$)}{\IF $n.izq$ = NULL THEN $\emptyset$ ELSE Ag($*$($n.izq$), hijos($*$($n.izq$))) FI \\
  $\cup$ \IF $n.der$ = NULL THEN $\emptyset$ ELSE Ag($*$($n.der$), hijos($*$($n.der$))) FI}
  ~
  \tadOperacion{nodos}{puntero(nodo)}{conj(nodo)}{}
  \tadAxioma{nodos($a$)}{\IF $a$ = NULL THEN $\emptyset$ ELSE Ag($*$($a$), hijos($*$($a$))) FI}
  ~
  \tadOperacion{altura}{puntero(nodo)}{nat}{}
  \tadAxioma{altura($a$)}{\IF $a = \NULL$ THEN 0 ELSE 1 + m\'ax(altura($a \to izq$), altura($a \to der$)) FI}
  ~
  \tadOperacion{padres}{puntero(nodo),conj(nodo)}{conj(nodo)}{}
  \tadAxioma{padres($a$, $ns$)}{\IF dameUno($ns$).$izq$ $=$ $a$ $\lor$ dameUno($ns$).$der$ $=$ $a$ THEN Ag(dameUno($ns$), padres($a$, sinUno($ns$)) ELSE padres($a$, sinUno($ns$)) FI}

\end{Representacion}

\begin{Algoritmos}

  \SetKwFunction{nil}{Nil}
  \SetKwFunction{bin}{Bin}
  \SetKwFunction{esNil}{EsNil}
  \SetKwFunction{raiz}{Raiz}
  \SetKwFunction{izq}{Izq}
  \SetKwFunction{der}{Der}
  \SetKwFunction{altura}{Altura}
  \SetKwFunction{cantNodos}{CantNodos}
  \SetKwFunction{inord}{Inorder}
  \SetKwFunction{preord}{Preorder}
  \SetKwFunction{postord}{Postorder}

  \begin{algoritmo}{iNil}{}{puntero(nodo)}
    $res \gets \NULL$ \com*{$\Theta(1)$}
  \end{algoritmo}
  \datosAlgoritmo{} % Descripción
  {} % Pre
  {} % Post
  {$\Theta(1)$} % Complejidad
  {} % Justificación

  \begin{algoritmo}{iBin}{\In{i}{puntero(nodo)}, \In{r}{$\alpha$}, \In{d}{puntero(nodo)}}{puntero(nodo)}
    $res \gets \new \langle r$, $i$, $d$, 1 + \maximo{\altura{$i$}, \altura{$d$}}, 1 + \cantNodos{$i$} + \cantNodos{$d$} $\rangle$ \tcp*[h]{Reservamos memoria para el nuevo nodo} \com*{$\Theta(1)$}
  \end{algoritmo}
  \datosAlgoritmo{} % Descripción
  {} % Pre
  {} % Post
  {$\Theta(1)$} % Complejidad
  {} % Justificación

  \begin{algoritmo}{iEsNil}{\In{a}{ab($\alpha$)}}{bool}
    $res \gets a = \NULL$ \com*{$\Theta(1)$}
  \end{algoritmo}
  \datosAlgoritmo{} % Descripción
  {} % Pre
  {} % Post
  {$\Theta(1)$} % Complejidad
  {} % Justificación

  \begin{algoritmo}{iRaiz}{\In{a}{ab($\alpha$)}}{$\alpha$}
    $res \gets a \to valor$ \com*{$\Theta(1)$}
  \end{algoritmo}
  \datosAlgoritmo{} % Descripción
  {} % Pre
  {} % Post
  {$\Theta(1)$} % Complejidad
  {} % Justificación

  \begin{algoritmo}{iIzq}{\In{a}{ab($\alpha$)}}{ab($\alpha$)}
    $res \gets a \to izq$ \com*{$\Theta(1)$}
  \end{algoritmo}
  \datosAlgoritmo{} % Descripción
  {} % Pre
  {} % Post
  {$\Theta(1)$} % Complejidad
  {} % Justificación

  \begin{algoritmo}{iDer}{\In{a}{ab($\alpha$)}}{ab($\alpha$)}
    $res \gets a \to der$ \com*{$\Theta(1)$}
  \end{algoritmo}
  \datosAlgoritmo{} % Descripción
  {} % Pre
  {} % Post
  {$\Theta(1)$} % Complejidad
  {} % Justificación

  % \begin{algoritmo}{Altura}{\In{a}{ab($\alpha$)}}{nat}
  %   $res \gets a \to alturaSubarbol$ \com*{$\Theta(1)$}
  % \end{algoritmo}
  % \datosAlgoritmo{} % Descripción
  % {} % Pre
  % {} % Post
  % {$\Theta(1)$} % Complejidad
  % {} % Justificación

  % \begin{algoritmo}{CantNodos}{\In{a}{ab($\alpha$)}}{nat}
  %   $res \gets a \to cantNodosSubarbol$ \com*{$\Theta(1)$}
  % \end{algoritmo}
  % \datosAlgoritmo{} % Descripción
  % {} % Pre
  % {} % Post
  % {$\Theta(1)$} % Complejidad
  % {} % Justificación

  \begin{algoritmo}{iAltura}{\In{a}{ab($\alpha$)}}{nat}
    \eIf(\com*[f]{$\Theta(1)$}){\esNil{$a$}}{
      $res \gets 0$ \com*{$\Theta(1)$}
    }{
      $res \gets$ 1 + \maximo{\altura{\izq{$a$}}, \altura{\der{$a$}}} \com*{$\Theta(n)$ (ver justificaci\'on)}
    }
  \end{algoritmo}
  \datosAlgoritmo{} % Descripción
  {} % Pre
  {} % Post
  {$\Theta(n)$} % Complejidad
  {Cada nodo interior del \'arbol llama a la funci\'on recursivamente sobre sus dos hijos, por lo que la funci\'on se ejecuta exactamente una vez por cada uno de los $n$ nodos del \'arbol. Como las operaciones que se realizan, sin contar la llamada recursiva, tienen complejidad $\Theta(1)$, la complejidad total resulta $\Theta(n)$.} % Justificación

  \begin{algoritmo}{iCantNodos}{\In{a}{ab($\alpha$)}}{nat}
    \eIf(\com*[f]{$\Theta(1)$}){\esNil{$a$}}{
      $res \gets 0$ \com*{$\Theta(1)$}
    }{
      $res \gets$ 1 + \cantNodos{\izq{$a$}} + \cantNodos{\der{$a$}} \com*{$\Theta(n)$ (ver justificaci\'on)}
    }
  \end{algoritmo}
  \datosAlgoritmo{} % Descripción
  {} % Pre
  {} % Post
  {$\Theta(n)$} % Complejidad
  {Cada nodo interior del \'arbol llama a la funci\'on recursivamente sobre sus dos hijos, por lo que la funci\'on se ejecuta exactamente una vez por cada uno de los $n$ nodos del \'arbol. Como las operaciones que se realizan, sin contar la llamada recursiva, tienen complejidad $\Theta(1)$, la complejidad total resulta $\Theta(n)$.} % Justificación

\end{Algoritmos}