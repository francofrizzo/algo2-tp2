\newcommand{\avlKS}{diccAVL(\ensuremath{\kappa}, \ensuremath{\sigma})} % Macro para diccAVL(k, s)

\section{M\'odulo Diccionario AVL($\kappa$, $\sigma$)}

\begin{Interfaz}
  
  \begin{paramFormales}
    \paramGeneros{$\kappa$, $\sigma$}

    \paramFuncion{
      \InterfazFuncion{$\bullet = \bullet$}{\In{k_1}{$\kappa$}, \In{k_2}{$\kappa$}}{\tipo{bool}}
      {$res \igobs (k_1 = k_2)$}
      [$\Theta(equal(k_1, k_2))$]
      [funci\'on de igualdad de $\kappa$]
    }

    \paramFuncion{
      \InterfazFuncion{$\bullet = \bullet$}{\In{k_1}{$\kappa$}, \In{k_2}{$\kappa$}}{\tipo{bool}}
      {$res \igobs (k_1 \leq k_2)$}
      [$\Theta(order(k_1, k_2))$]
      [funci\'on de comparaci\'on por orden total estricto de $\kappa$]
    }

    \paramFuncion{
      \InterfazFuncion{Copiar}{\In{k}{$\kappa$}}{$\kappa$}
      {$res \igobs k$}
      [$\Theta(copy(k))$]
      [funci\'on de copia de $\kappa$]
    }

    \paramFuncion{
      \InterfazFuncion{Copiar}{\In{s}{$\sigma$}}{$\sigma$}
      {$res \igobs s$}
      [$\Theta(copy(s))$]
      [funci\'on de copia de $\sigma$]
    }
  \end{paramFormales}

  \seExplicaCon{Diccionario($\kappa$, $\sigma$)}

  \generos{\tipo{\avlKS}}

  \Encabezado{Operaciones de diccionario}

    \InterfazFuncion{Vacio}{}{\avlKS}
    {$res \igobs$ vac\'io}
    []
    []

    \InterfazFuncion{Definir}{\Inout{d}{\avlKS}, \In{k}{$\kappa$}, \In{s}{$\sigma$}}{}
    [$d \igobs d_0$]
    {$d \igobs$ definir($k$, $s$, $d_0$)}
    []
    []

    \InterfazFuncion{Borrar}{\Inout{d}{\avlKS}, \In{k}{$\kappa$}}{}
    [$d \igobs d_0 \land$ def?($k$, $d$)]
    {$d \igobs$ borrar($k$, $d_0$)}
    []
    []

    \InterfazFuncion{\#Claves}{\In{d}{\avlKS}}{nat}
    {$res \igobs$ \#(claves($d$))}
    []
    []

    \InterfazFuncion{Definido?}{\In{d}{\avlKS}, \In{k}{$\kappa$}}{bool}
    {$res \igobs$ def?($k$, $d$)}
    []
    []

    \InterfazFuncion{Obtener}{\In{d}{\avlKS}, \In{k}{$\kappa$}}{$\sigma$}
    [def?($k$, $d$)]
    {$res \igobs$ obtener($k$, $d$)}
    []
    []

\end{Interfaz}

\begin{Representacion}

  \begin{Estructura}{diccAVL($\alpha$)}[estrAVL]

    \begin{Tupla}[estrAVL]
      \tupItemNL{raiz}{puntero(nodo)}%
      \tupItemNL{cantNodos}{nat}%
    \end{Tupla}

    \begin{Tupla}[nodo]
      \tupItemNL{clave}{$\kappa$}%
      \tupItemNL{significado}{$\sigma$}%
      \tupItemNL{padre}{puntero(nodo)}%
      \tupItemNL{izq}{puntero(nodo)}%
      \tupItemNL{der}{puntero(nodo)}%
      \tupItemNL{altSubarbol}{nat}%
    \end{Tupla}

  \end{Estructura} 

  \Rep[estrAB][ab]{
    ($ab$.cantNodos = 0) = ($ab$.raiz = NULL) $\yluego$ \\
    ($ab$.cantNodos > 0) $\impluego$ ($ab$.cantNodos = cantHijos(\*($ab$.raiz)) + 1) $\land$ \\
    (\paratodo{nodo}{n_1, n_2})(($n_1$ $\in$ nodos($ab$) $\land$ $n_2$ $\in$ nodos($ab$)) $\impluego$ (($n_1$.izq $\neq$ $n_2$.der) $\land$ (($n_1$.izq = $n_2$.izq $\lor$ $n_1$.der = $n_2$.der) $\implies$ ($n_1 = n_2$))))
  }

  \AbsFc[estrAB]{ab($\alpha$)}[ab]{
    \IF $ab$.cantNodos = 0 THEN nil ELSE bin(Abs(plantar($ab$.raiz $\to$ izq)), $ab$.raiz $\to$ dato, Abs(plantar($ab$.raiz $\to$ der))) FI
  }

  \tadOperacion{hijos}{nodo}{conj(nodo)}{}
  \tadAxioma{hijos($n$)}{\IF $n$.izq = NULL THEN $\emptyset$ ELSE Ag(\*($n$.izq), hijos(\*($n$.izq))) FI $\cup$ \IF $n$.der = NULL THEN $\emptyset$ ELSE Ag(\*($n$.der), hijos(\*($n$.der))) FI}

  \tadOperacion{cantHijos}{nodo}{nat}{}
  \tadAxioma{cantHijos($n$)}{\#(hijos($n$))}

  \tadOperacion{nodos}{estrAB}{conj(nodo)}{}
  \tadAxioma{nodos($ab$)}{\IF $ab$.raiz = NULL THEN $\emptyset$ ELSE Ag(\*($ab$.raiz), hijos(\*($ab$.raiz))) FI}

  \tadOperacion{plantar}{puntero(nodo)}{estrAB}{}
  \tadAxioma{plantar($p$)}{$\langle p$, \IF $p$ = NULL THEN 0 ELSE cantHijos(\*($p$)) + 1 FI$\rangle$}

\end{Representacion}

\begin{Algoritmos}
  
  % Definición de funciones que se usan en el algoritmo:
  \SetKwFunction{atajoFuncion}{nombreFuncion}

  \begin{algoritmo}{nombreAlgoritmo}{parametros}{tipoSalida}
    % Crear e inicializar variables:
    \tipo{tipoVariable} $nombreVariable \gets valor$ \;
    % Nota: las líneas sin comentario deben terminar en \;

    % Comentarios:
    Contenido de la linea \tcp*[h]{Comentario pegado al texto, onda C++} \com*{Comentario contra el margen}
    Contenido de la linea \com*[h]{Comentario pegado al texto} \tcp*{Comentario contra el margen, onda C++}

    % Llamada a función definida antes:
    \atajoFuncion{parametros}

    % If/then:
    \If{guarda}{
      Codigo \;
    }

    % If/then/else:
    \eIf(\com*[f]{Comentario opcional}){guarda}{
      Codigo (then) \;
    }(\com*[f]{Otro comentario opcional}){
      Codigo (else) \;
    }

    % If/then/elseif:
    \uIf{guarda}{
      Codigo (then) \;
    }\ElseIf{otra guarda}{
      Codigo (else) \;
    }

    % While:
    \While(\com*[f]{Comentario opcional}){guarda}{
      Codigo \;
    }

  \end{algoritmo}

  \SetKwFunction{fdb}{FDB}
  \SetKwFunction{raizq}{RotarAIzquierda}
  \SetKwFunction{rader}{RotarADerecha}
  \SetKwFunction{buscar}{Buscar}
  \SetKwFunction{max}{max}
  \SetKwFunction{rebalancear}{RebalancearArbol}

  \begin{algoritmo}{iVacio}{}{estrAVL}{
    $res \gets \langle$NULL, 0$\rangle$ \com*{$\Theta(1)$}
  }
  \end{algoritmo}

  \begin{algoritmo}{iRebalancearArbol}{\In{n}{puntero(nodo)}}{}
    \tipo{puntero(nodo)} $p \gets n$ \com*{$\Theta(1)$}
    \While{$p \neq \NULL$}{ 
      \tipo{int} $fdb1 \gets$ \fdb{$p$}\;
      \uIf{$fdb1 = 2$}{
        \tipo{puntero(nodo)} $q \gets (p \to der)$ \;
        \tipo{int} $fdb2 \gets$ \fdb{$q$} \;
        \uIf{$fdb2 = 1 \lor fdb2 = 0$}{
          \raizq{$p$} \;
          $p \gets q$ \;
        }\ElseIf{$fbd2 = -1$}{
          \rader{$q$} \;
          \raizq{$p$} \;
          $p \gets (q \to padre)$ \;
        }
      }\ElseIf{$fdb1 = -2$}{
        \tipo{puntero(nodo)} $q \gets (p \to izq)$ \;
        \tipo{int} $fdb2 \gets$ \fdb{$q$} \;
        \uIf{$fdb2 = -1 \lor fdb2 = 0$}{
          \rader{$p$} \;
          $p \gets q$ \;
        }\ElseIf{$fbd2 = 1$}{
          \raizq{$q$} \;
          \rader{$p$} \;
          $p \gets (q \to padre)$ \;
        }
      }
      $p \gets (p \to padre)$ \;
    }
  \end{algoritmo}

  \begin{algoritmo}{iFBD}{\In{n}{puntero(nodo)}}{int}
    \tipo{int} $altIzq \gets n \to izq = \NULL$ ? 0 : $n \to izq \to altSubarbol$ \;
    \tipo{int} $altDer \gets n \to der = \NULL$ ? 0 : $n \to izq \to altSubarbol$ \;
    $res \gets altDer - altIzq$ \;
  \end{algoritmo}

  \begin{algoritmo}{iRotarAIzquierda}{\In{n}{puntero(nodo)}}{}
    \If{$n.padre \neq \NULL$}{
      \eIf{$n.padre \to izq = n$}{
        $(n.padre \to izq) \gets n.der$ \;
      }{
        $(n.padre \to der) \gets n.der$ \;
      }
    }
    $(n.der \to padre) \gets n.padre$ \; 
    $n.padre \gets n.der$ \;
    $n.der \gets (n.der \to izq)$ \;
    \If{$n.der \neq \NULL$}{
      $(n.der \to padre) \gets n$ \;
    }
    $(n.padre \to izq) \gets n$ \;
    \While{$actual \neq \NULL$}{
      $(actual \to altSubarbol) \gets 1$ + \max{$actual \to izq \to altSubarbol$, $actual \to der \to altSubarbol$} \;
      $actual \gets (actual \to padre)$ \;
    }
  \end{algoritmo}

  \begin{algoritmo}{iRotarADerecha}{\In{n}{puntero(nodo)}}{}
    \If{$n.padre \neq \NULL$}{
      \eIf{$n.padre \to izq = n$}{
        $(n.padre \to izq) \gets n.izq$ \;
      }{
        $(n.padre \to der) \gets n.izq$ \;
      }
    }
    $(n.izq \to padre) \gets n.padre$ \; 
    $n.padre \gets n.izq$ \;
    $n.izq \gets (n.izq \to der)$ \;
    \If{$n.izq \neq \NULL$}{
      $(n.izq \to padre) \gets n$ \;
    }
    $(n.padre \to der) \gets n$ \;
    \While{$actual \neq \NULL$}{
      $(actual \to altSubarbol) \gets 1$ + \max{$actual \to izq \to altSubarbol$, $actual \to der \to altSubarbol$} \;
      $actual \gets (actual \to padre)$ \;
    }
  \end{algoritmo}

  \begin{algoritmo}{iBuscar}{\In{e}{estrAVL}, \In{k}{$\kappa$}, \Out{padre}{puntero(nodo)}}{puntero(nodo)}
    $padre \gets \NULL$ \;
    $actual \gets e.raiz$ \;
    \While{$actual \neq \NULL \yluego (actual \to clave \neq k)$}{ % ¡Preguntar por el y luego!
      $padre \gets actual$ \;
      \eIf{$k \leq (padre \to clave)$}{
        $actual \gets (actual \to izq)$ \;
      }{
        $actual \gets (actual \to der)$ \;
      }
    }
    $res \gets actual$ \;
    \end{algoritmo}

    \begin{algoritmo}{iBorrar}{\Inout{e}{estrAVL}, \In{k}{$\kappa$}}{}
      \tipo{puntero(nodo)} $padre \gets \NULL$ \;
      \tipo{puntero(nodo)} $lugar \gets$ \buscar{$e$, $k$, $padre$} \;
      \uIf{$lugar \to izq = \NULL \land lugar \to der = \NULL$}{
        \eIf{$padre \neq \NULL$}{
          \eIf{$padre \to izq = lugar$}{
            $(padre \to izq) \gets \NULL$ \; 
            $(padre \to altSubarbol) \gets 1 + (padre \to der \to altSubarbol)$ \;
          }{
            $(padre \to der) \gets \NULL$ \; 
            $(padre \to altSubarbol) \gets 1 + (padre \to izq \to altSubarbol)$ \;
          }
          \rebalancear{$padre$} \;
        }{
          $e.raiz = \NULL$ \;
        }
        \delete{$lugar$} \;
      }\uElseIf{$lugar \to der = \NULL$}{
        $(lugar \to izq \to padre) \gets padre$ \;
        \eIf{$padre \neq \NULL$}{
          \eIf{$padre \to izq = lugar$}{
            $(padre \to izq) \gets (lugar \to izq)$ \; 
          }{
            $(padre \to der) \gets (lugar \to izq)$ \; 
          }
          $(padre \to altSubarbol) \gets 1$ + \max{$padre \to der \to altSubarbol$, $padre \to izq \to altSubarbol$} \;
          \rebalancear{$padre$} \;
        }{
          $e.raiz = lugar \to izq$ \;
        }
        \delete{$lugar$} \;
      }\uElseIf{$lugar \to izq = \NULL$}{
        $(lugar \to der \to padre) \gets padre$ \;
        \eIf{$padre \neq \NULL$}{
          \eIf{$padre \to izq = lugar$}{
            $(padre \to izq) \gets (lugar \to der)$ \; 
          }{
            $(padre \to der) \gets (lugar \to der)$ \; 
          }
          $(padre \to altSubarbol) \gets 1$ + \max{$padre \to der \to altSubarbol$, $padre \to izq \to altSubarbol$} \;
          \rebalancear{$padre$} \;
        }{
          $e.raiz = lugar \to izq$ \;
        }
        \delete{$lugar$} \;
      }\Else{
        \com*{Ac\'a viene la parte horrible del algoritmo}
      }
    \end{algoritmo}

\end{Algoritmos}