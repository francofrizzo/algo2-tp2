\section{M\'{o}dulo DCNet}

\Encabezado{Notas preliminares}
  En todos los casos, al indicar las complejidades de los algoritmos, las variables que se utilizan corresponden a:
  \vspace{-0.5em}\begin{itemize}
    \item $n$: N\'umero de computadoras en la red.
    \item $k$: Longitud de la cola de paquetes m\'as larga al momento.
    \item $L$: Longitud de nombre de computadora m\'as largo de la red.
    \item $i$: Mayor cantidad de interfaces que tiene alguna computadora en la red en el momento.
  \end{itemize}

\servUsados{diccLog, Heap, Red, diccStr, tupla, conjunto }

\begin{Interfaz}
  
  \seExplicaCon{DCNet}
  
  \generos{\tipo{dcnet}}
  
  \Encabezado{Operaciones de DCNet} % ¿?

  \InterfazFuncion{iniciarDCNet}{\In{r}{Red}}{dcnet}%
  [true]%pre
  {res $\igobs$ iniciarDCNet(r)}%pos
  [$\Theta(n^3 \times I^2 \times L + n^2 \times n!)$]%complejidad
  [Genera un nuevo DCNet sin paquetes. ]%descripcion
  []%aliasing
  
  \InterfazFuncion{crearPaquete}{\Inout{D}{DCNet}, \In{p}{paquete}}{}%
  [D $\igobs$ $D_{0}$ $\land$ $\not$(($\exists p'$: paquete)(paqueteEnTransito?(D,p') $\land$ id(p') = id(p)) $\land$ origen(p) $\in$ computadoras(red(D)) $\yluego$ destino(p) $\in$ computadoras(red(D)) $\yluego$ hayCamino?(red(D), origen(p), destino(p)))]%pre
  {D $\igobs$ crearPaquete($D_{0}$, p)}%pos
  [$\Theta(L+log(k))$]%complejidad
  [Crea un nuevo paquete que no exite en el DCNet anterior. ]%descripcion
  []%aliasing

  \InterfazFuncion{avanzarSegundo}{\Inout{D}{DCNet}}{}%
  [D $\igobs$ $D_{0}$]%pre
  {D $\igobs$ avanzarSegundo($D_{0}$)}%pos
  [$\Theta(n \times log(k))$]%complejidad
  [Avanza un segundo en el DCNet, moviendo todos los paquetes correspondientes. ]%descripcion
  []%aliasing

  \InterfazFuncion{red}{\In{D}{DCNet}}{red}%
  [true]%pre
  {esAlias(res $\igobs$ red(D))}%pos
  [$\Theta(1)$]%complejidad
  [Devuelve la red donde esta funcionando el DCNet. ]%descripcion
  [Pasamos la red por referencia]%aliasing

  \InterfazFuncion{caminoRecorido}{\In{D}{DCNet}, \In{p}{paquete}}{secu(compu)}%
  [paqueteEnTransito?(D, p)]%pre
  {res $\igobs$ caminoRecorrido(D, p)}%pos
  [$\Theta(n \times log(k))$]%complejidad
  [Devuelve la secuencia que contiene de forma ordenada todas las computadoras por las que fue pasando. ]%descripcion
  []%aliasing

  \InterfazFuncion{cantidadEnviados}{\In{D}{DCNet}, \In{c}{compu}}{nat}%
  [c $\in$ computadoras(red(D))]%pre
  {res $\igobs$ cantidadEnviados(D, c)}%pos
  [$\Theta(L)$]%complejidad
  [Devuelve la cantidad de paquetes que envi\'o la computadora ``c''. ]%descripcion
  []%aliasing

  \InterfazFuncion{enEspera}{\In{D}{DCNet}, \In{c}{compu}}{conj(paquete)}%
  [c $\in$ computadoras(red(D))]%pre
  {esAlias(res $\igobs$ enEspera(D, c))}%pos
  [$\Theta(L)$]%complejidad
  [Devuelve los paquetes que tiene en espera la compu ``c''. ]%descripcion
  [El conjunto de paquetes en espera se devuelve por referencia]%aliasing

  \InterfazFuncion{paqueteEnTr\'ansito?}{\In{D}{DCNet}, \In{p}{paquete}}{bool}%
  [true]%pre
  {res $\igobs$ paqueteEnTransito?(D, p)}%pos
  [$\Theta(n \times log(k))$]%complejidad
  [Devuelve ``True'' si y solo si el paquete esta en los paquetes en espera de alguna computadora. ]%descripcion
  []%aliasing

  \InterfazFuncion{laQueM\'asEnvi\'o}{\In{D}{DCNet}}{compu}%
  [true]%pre
  {res $\igobs$ laQueM\'asEnvi\'o(D)}%pos
  [$\Theta(1)$]%complejidad
  [Devuelve una de las computadoras con mas paquetes enviados]%descripcion
  []%aliasing

%  \InterfazFuncion{NOMBRE}{PARAMTROS}{RES}%
%  []%pre
%  {}%pos
%  []%complejidad
%  []%descripcion
%  []%aliasing

\end{Interfaz}
\pagebreak
\begin{Representacion}

  \begin{Estructura}{dcnet}[estrDCNet]

    \begin{Tupla}[estrDCNet]
      \tupItemNL{red}{red}
      \tupItemNL{IDsCompusPorIP}{diccStr(string, nat)}
      \tupItemNL{siguientesCompus}{ad(ad(nat))}
      \tupItemNL{paquetesEnEspera}{ad(tupla( \\
        \hspace*{3mm}\campoTupla{enConjunto}{conj(paquete)}, \\
        \hspace*{3mm}\campoTupla{porID}{diccLog(ID, tupla(\campoTupla{iPaquete}{itConj(paquete)}, \campoTupla{codOrigen}{nat}, \campoTupla{codDestino}{nat})}, \\
        \hspace*{3mm}\campoTupla{porPrioridad}{colaPrior(tupla(prioridad, itConjunto(paquete))))} \\
      ))}
      \tupItemNL{\#PaqEnviados}{ad(nat)}
      \tupItemNL{laQueM\'asEnvi\'o}{compu}
      \tupItemNL{IPsCompuPorID}{ad(compu)}
    \end{Tupla}

  \end{Estructura} 

\textbf{Invariante de representaci\'on en castellano:}

    \begin{enumerate} 
      \item Tam(PaquetesEnEspera) $=$ Tam($\#$PaquetesEnviados) $=$ Tam(ProximaCompu) $=$ Tam(IPsCompusPorID) $=$ $\#$(computadoras(red)). Ademas, la longitud de cada uno de los arreglos de ProximaCompu es igual a \#(computadoras(red)).
      \item La que mas envio, es la computadora correspondiente al maximo del arreglo \#PaquetesEnviados.
      \item Los significados del diccStr IDsCompusPorIP, son consecutivos, iniciando en 0 hasta n.
      \item compus(red) $=$ claves(diccStr)
      \item Para todas las posiciones del arreglo paquetesEnEspera, se debe cumplir lo siquiente:
              \begin{enumerate} 
              \item Los iteradores del AVL tienen siguiente.
              \item Los iteradores del heap tienen siguiente
              \item Todos los iteradores del AVL, apuntan a un paquete que se encuentra en el conjunto.
              \item Todas las claves del AVL, son las IDs de los paquetes apuntados por el iterador que se encuentra en sus significados.
              \item Todos los iteradores del heap, apuntan a un paquete que se encuentra en el conjunto.
              \item Las prioridades de los paquetes apuntados por los iteradores del heap, se corresponden con la prioridad indicada por el heap.
              \item No hay dos iteradores en el heap apuntando al mismo paquete.
              \item La cantidad de paquetes en el conjunto, en el heap y en el AVL es la misma.
              \item En los significados del AVL, el origen y el destino, se corresponden con el origen y el destino del paquete apuntado por el iterador.
              \end{enumerate}
      \item No hay paquetes repetidos entre las distintas posiciones de paquetesEnEspera.
      \item Para cada computadora en el arreglo IPsCompusPorID, usando su IP como clave en el diccStr IDsCompusPorIP, se obtiene como significado la posicion donde se encuentra (en el arreglo).
      \item Para todo (i,j) que esten definidos en siguientesCompus, siguientesCompus[i][j] es la primer compu de alguno de los caminos minimos para llegar desde i a j.
      \item Todas las computadoras del arreglo IPsCompusPorID son las mismas que las computadoras de red.
      \item No hay repetidos en el arreglo IPsCompusPorID


    \end{enumerate}



	\Rep[e][l]{
	($\forall e$: estrDCNet)(
\begin{enumerate} 
%1
\item (tam($e$.paquetesEnEspera) = tam($e$.IPcompusPorID) = tam($e$.siguienteCompu) = tam($e$.\#PaqEnviados) = \#(computadoras($e$.estrRed))) $\land$ 
		($\forall n$: nat)(definido?($e$.siguienteCompu,n) $\impluego$ tam($e$.siguienteCompu[n]) = \#(computadoras($e$.estrRed))) $\land$

%2
\item IPsCompusPorID[PosMaxima($e$.\#PaqEnviados)] = $e$.laQueM\'asEnvi\'o $\land$
%3 
\item ($\forall c$: compu)(c $\in$ computadoras($e$.estrRed) $\impluego$ obtener($\Pi_1$(c),$e$.IDcompusPorIP) < \#(computadoras($e$.estrRed))) $\land$
		(($\forall c_1$, $c_2$: compu)(($c_1 \in$ computadoras($e$.estrRed) $\land$ ($c_2 \in$ computadoras($e$.estrRed)) $\land$ ($c_1 \neq c_2$)) $\impluego$ ((obtener($\Pi_1$($c_1$), $e$.IDcompusPorIP) $\neq$ obtener($\Pi_1$($c_2$), $e$.IDcompusPorIP))))) $\land$

%4
\item(dameNombres(computadoras($e$.estrRed)) = claves(IDcompusPorIP)) $\land$

%5
\item	($\forall L$: nat)( 0 $\leq$ $L$ < tam($e$.paqEnEspera) $\impluego$ (
 \begin{enumerate}

      \item ($\forall$ $it_1$: ItConj(paquete)) $it$ $\in$ dame$\Pi_1$(juntarSignificados($\Pi_2$($e$.paquetesEnEspera[$L$]))) $\impluego$ haySiguiente?($it_1$) $\land$ 

      \item ($\forall$ $it_2$: ItConj(paquete)) $it_2$ $\in$ dame$\Pi_2$(colaAConj($\Pi_3$($e$.paquetesEnEspera[$L$]))) $\impluego$ haySiguiente?($it_2$) $\yluego$ 

			\item ($\forall i$: ItConj(paquete))(siguiente(i) $\in$ (dameSiguientes(dame$\Pi_1$(juntarSignificados$_1$( $\Pi_2$($e$.paquetesEnEspera[$L$])))$\impluego$ siguiente(i) $\in$ $\Pi_1$($e$.paquetesEnEspera[$L$])))) $\land$ 

			\item ($\forall c$: ID) (c $\in$ claves($\Pi_2$($e$.paquetesEnEspera[$L$]))) $\impluego$ $\Pi_1$(siguiente($\Pi_1$(obtener(c, $\Pi_2$($e$.paquetesEnEspera[$L$]))))) = c $\land$ 

      \item ($\forall$ $it$:ItConj(paquete)) siguiente($it$) $\in$ dameSiguientes(dame$\Pi_2$(colaAConj( $\Pi_3$($e$.paquetesEnEspera[$L$])))) $\impluego$ siguiente($it$) $\in$ $\Pi_1$($e$.paquetesEnEspera[$L$]) 

			\item ($\forall$ $t$: tupla(prioridad,ItConj(paquete))) $t$ $\in$ colaAConj($\Pi_3$($e$.paquetesEnEspera[$L$])) $\Rightarrow$ siguiente($\Pi_2$($t$)).prioridad = $\Pi_1$($t$) 

      \item ($\forall$ $it_{1}$: ItConj(paquete)) ($\forall$ $it_{2}$: ItConj(paquete)) $it_{1}$ $\in$ dame$\Pi_{2}$(colaAConj($\Pi_{3}$($e$.paquetesEnEspera[$L$]))) $\land$ $it_{2}$ $\in$ dame$\Pi_{2}$(colaAConj( $\Pi_{3}$($e$.paquetesEnEspera[$L$]))) $it_{1}$ $\neq$ $it_{2}$ $\impluego$ siguiente($it_{1}$) $\neq$ siguiente($it_{2}$) 

      \item \#($\Pi_1$($e$.paquetesEnEspera[L])) = \#claves($\Pi_2$($e$.paquetesEnEspera[L])) $\land$ = \#colaAConj($\Pi_3$($e$.paquetesEnEspera[L])) 

      \item ($\forall c$:ID) $c \in$ claves($\Pi_2$($e$.paquetesEnEspera) $\impluego$ obtener(siguiente($\Pi_1$(obtener($c$, $\Pi_2$($e$.paquetesEnEspera))).origen, $e$.IDsCompusPorIP) = $\Pi_2$(obtener($c$, $\Pi_2$($e$.paquetesEnEspera)))) $\land$ obtener(siguiente($\Pi_1$(obtener($c$, $\Pi_2$($e$.paquetesEnEspera))).destino, $e$.IDsCompusPorIP) = $\Pi_3$(obtener($c$, $\Pi_2$($e$.paquetesEnEspera))))

  \end{enumerate}
%6
\item     ($\forall x$, $z$: nat)((0 $\leq$ x < tam($e$.paquetesEnEspera) $\land$ 0 $\leq$ z < tam($e$.paquetesEnEspera) $\land$ x $\neq$ z) $\impluego$ ($\Pi_1$($e$.paquetesEnEspera[x]) $\cap$ $\Pi_1$($e$.paquetesEnEspera[z])) = $\emptyset$) $\land$ 

%7
\item	($\forall i$: nat)(0 $\leq$ i < \#(computadoras($e$.estrRed)) $\impluego$ obtener($e$.IPcompusPorID[$i$].IP, $e$.IDcompusPorID) = i) $\land$ 

%8
\item	($\forall n$, $m$: nat)(0 $\leq$ n < \#(computadoras($e$.estrRed)) $\land$ 0 $\leq$ m < \#(computadoras($e$.estrRed)) $\impluego$ ($\exists$ $x$: (secu(compu))) $x$ $\in$ caminosminimos($e$.estrRed, $e$.IPcompusPorID[n], $e$.IPcompusPorID[m]) $\land$ prim($x$) $=$ $e$.siguienteCompu[n][m] 

%9
\item    ($\forall i$: nat)(0 $\leq$ i $\leq$ \#computadoras($e$.estrRed) $\impluego$ $e$.IPcompusPorID[$i$] $\in$ computadoras($e$.estrRed)) $\land$ 

%10
\item    ($\forall x$, $y$: nat)((0 $\leq$ x < \#computadoras($e$.estrRed) $\land$ 0 $\leq$ y < \#computadoras($e$.estrRed) $\land$ $x$ $\neq$ y) $\impluego$ $e$.IPcompusPorID[x] $\neq$ $e$.IPcompusPorID[y]) $\land$ 


    \end{enumerate} 
    
    }\mbox{} %habia un parentesis no se de donde sale (estaba cerrando en esta linea)

\pagebreak

\AbsFc[estrDCNet]{DCNet}[e]{red($d$) = $e$.Red $\land$ \\

  ($\forall p$: paquete) paqueteEnTr\'ansito?($d$, $p$) $\impluego$ caminoRecorrido($d$, $p$) = caminoDelPaquete($e$.siguienteCompu, $e$.IPsCompusPorID, $e$.PaquetesEnEspera, $p$, obtener($e$.IDsCompusPorIP, $p$.origen.IP), obtener($e$.IDsCompusPorIP, $p$.destino.IP)) $\land$ \\

  ($\forall c$: compu) $c$ $\in$ computadoras(red($d$)) $\impluego$ cantidadEnviados($d$, $c$) = $e$.\#PaqEnviados[obtener($c$.IP, $e$.IDsCompusPorIP)] $\land$ \\

  ($\forall c$: compu) $c$ $\in$ computadoras(red($d$)) $\impluego$ enEspera($d$, $c$) = enConjunto($e$.paquetesEnEspera[obtener($c$.IP, $e$.IDsCompusPorIP)])
}

  ~

  \textbf{Funciones Auxiliares:}

  ~

\tadTipoFuncion{ad(ad(nat)), ad(compu), ad(tupla(enConjunto: conj(paquete){,} porID: diccLog, tupla(iPaquete: itConj(paquete){,} codOrigen: nat{,} codDestino: nat{,} porPrioridad: colaPrior(tupla(prioridad{,} itConjunto(paquete)))))), paquete, nat, nat}

\tadOperacion{caminoDelPaquete}{ad(ad(nat)), ad(compu), ad(tupla(enConjunto: conj(paquete){,} porID: diccLog, tupla(iPaquete: itConj(paquete){,} codOrigen: nat{,} codDestino: nat{,} porPrioridad: colaPrior(tupla(prioridad{,} itConjunto(paquete)))))), paquete, nat, nat}{secu(compu)}{$(\exists L$:nat) 0 $\leq$ L < tam($e$.paquetesEnEspera) $\yluego$ def?($p$.ID,  $\Pi_2$($e$.paquetesEnEspera[L]))}   

\small\begin{verbatim}
\caminoDelPaquete: ad(ad(nat)), ad(compu), ad(tupla(enConjunto: conj(paquete), 
porID: diccLog, tupla(iPaquete: itConj(paquete),codOrigen: nat, 
 codDestino: nat, porPrioridad: colaPrior(tupla(prioridad, 
 itConjunto(paquete)))))), paquete, nat, nat -> secu(compu)  
\end{verbatim}

\tadAxioma{caminoDelPaquete($t$, $CsxID$, $ps$, $p$, $compuActual$, $d$)}{\IF def?(ID($p$), porID($ps$[$compuActual$])) THEN $CsxID$[$compuActual$] $\puntito$ <> ELSE $CsxID$[$compuActual$] $\puntito$ caminoDelPaquete($t$, $CsxID$, $ps$, $p$, $t$[$compuActual$][$d$], $d$) FI}

  ~

  \tadOperacion{colaAConj}{colaPrior(tupla<prioridad{,} ItConj(paquete)>)}{conj(tupla<prioridad{,} ItConj(paquete)>)}{}
  \tadAxioma{colaAConj($c$)}{\IF vacia?(c) THEN $\emptyset$ ELSE ag(proximo(c), colaAConj(desencolar(c))) FI}

  ~
  
  \tadOperacion{dame$\Pi_{1}$}{conj(tupla(itConj(paquete), nat, nat))}{conj(ItConj(paquete))}{}
  \tadAxioma{dame$\Pi_{1}$($c$)}{\IF $\emptyset$?(c) THEN $\emptyset$ ELSE ag($\Pi_{1}$(dameuno(c)), dame$\Pi_{1}$(sinuno(c))) FI}

  ~

  \tadOperacion{dame$\Pi_{2}$}{conj(tupla<prioridad, ItConj(paquete)>)}{conj(ItConj(paquete))}{}
  \tadAxioma{dame$\Pi_{2}$($c$)}{\IF $\emptyset$?(c) THEN $\emptyset$ ELSE ag($\Pi_{2}$(dameuno(c)), dame$\Pi_{2}$(sinuno(c))) FI}

  ~
  
  \tadOperacion{dameSiguientes}{conj(ItConj(paquete))}{conj(paquete)}{}
  \tadAxioma{dameSiguientes($c$)}{\IF $\emptyset$?(c) THEN $\emptyset$ ELSE ag(siguiente(dameuno(c)), dameSiguientes(sinuno(c))) FI}

  ~

  \tadOperacion{PosMaxima}{ad(nat)/a}{nat}{tam($a$)>0}
  \tadAxioma{PosMaxima($a$)}{Posicion(Maximo($a$, $a$[0]), $a$)}

  ~

  \tadOperacion{Maximo}{ad(nat), nat}{nat}{}
  \tadAxioma{Maximo($a$, $n$)}{\IF (tam($a$)$=0$) THEN n ELSE {\IF ($a$[0]>$n$) THEN Maximo(finAd(a), a[0]) ELSE Maximo(finAd(a), n) FI} FI}

  ~

  \tadOperacion{Posicion}{nat, ad(nat)}{nat}{definido?($a$,$n$)}
  \tadAxioma{Posicion($n$, $a$)}{Posicion\fAux($a$, $n$, 0)}

  ~

  \tadOperacion{Posicion\fAux}{ad(nat), nat, nat}{nat}{}
  \tadAxioma{Posicion\fAux($a$, $n$, $i$)}{\IF ($a$[0]=$n$) THEN $i$ ELSE Posicion\fAux(finAd($a$), $n$, $i+1$) FI}

  ~

  \tadOperacion{dameNombres}{conj(compu)}{conj(IP)}{}
  \tadAxioma{dameNombres($c$)}{\IF $\emptyset$?($c$) THEN $\emptyset$ ELSE Ag(dameUno($c$.IP, dameNombres(sinUno($c$)))) FI}

  ~

  \tadOperacion{JuntarSignificados}{dicc(IP{,} tupla(itConj(paquete){,} nat{,} nat))}{conj(tupla(itConj(paquete), nat, nat))}{}
  \tadAxioma{JuntarSignificados($d$)}{JuntarSignificados\fAux($d$, claves($d$))}

  ~

  \tadOperacion{JuntarSignificados\fAux}{dicc(IP{,} tupla(itConj(paquete){,} nat{,} nat)), conj(IP)}{conj(tupla(itConj(paquete){,} nat{,} nat))}{}
  \tadAxioma{JuntarSignificados\fAux($d$, $c$)}{\IF $\emptyset$?($c$) THEN $\emptyset$ ELSE Ag(obtener(dameUno($c$), $d$), JuntarSignificados\fAux($d$, sinUno($c$))) FI}

  ~

  \tadOperacion{colaAConj}{colaPrior(tupla(Prioridad{,} itConj(paquete)))}{conj(tupla(prioridad, itConj(paquete)))}{}
  \tadAxioma{colaAConj($c$)}{\IF vacia?($c$) THEN $\emptyset$ ELSE Ag(proximo($c$), colaAConj(desencolar($c$))) FI}

\end{Representacion}

\begin{Algoritmos}

  \SetKwFunction{crearArr}{CrearArreglo}
  \SetKwFunction{crearItRed}{CrearItRed}
  \SetKwFunction{card}{Cardinal}
  \SetKwFunction{compus}{Computadoras}

  \begin{algoritmo}{iIniciarDCNet}{\In{r}{Red}}{DCNet}
    $res$.red $\gets$ \copiar{$r$}\com*{$\Theta(n \times I \times L)$}   
    $res$.\#PaqEnviados $\gets$ \crearArr{cantCompus($res$.red)} \com*{$\Theta(n)$}   
    $res$.IPsCompuPorID $\gets$ \crearArr{cantCompus($res$.red)} \com*{$\Theta(n)$}   
    $res$.siguientesCompus $\gets$ \crearArr{cantCompus($res$.red)} \com*{$\Theta(n)$}   
    $res$.paquetesEnEspera $\gets$ \crearArr{cantCompus($res$.red)} \com*{$\Theta(n)$}   
    \tipo{conj(compu)} $c \gets$ computadoras(r) \com*{$\Theta(1)$}  
    \tipo{itConj} $it_{1} \gets$ crearItConj($c$) \com*{$\Theta(1)$}   
    \tipo{nat} $j \gets$ 0\com*{$\Theta(1)$}   
    \While(\com*[f]{$\Theta(n)$} iteraciones){$j$ < \card{\compus{$res$.red}}}{
      $res$.siguientesCompus[$j$] $\gets$ \crearArr{cantCompus($res$.red)} \com*{$\Theta(n)$}   
      $res$.\#PaqEnviados[$j$] $\gets$ 0\com*{$\Theta(1)$}   
      $res$.paquetesEnEspera[$j$] $\gets$ $\langle$vac\'{i}o(),vac\'{i}o(),vac\'{i}o()$\rangle$\com*{$\Theta(1)$}   
      definir(siguiente($it_{1}$).IP, $j$, $res$.IDsCompusPorIP)\com*{$\Theta(L)$}   
      $res$.IPsCompusPorID[$j$] $\gets$ siguiente($it_{1}$) \com*{$\Theta(1)$}   
      $j \gets j + 1$\com*{$\Theta(1)$}   
      avanzar($it_{1}$)\com*{$\Theta(1)$}   
    }
    \tipo{nat} $k \gets$ 0\com*{$\Theta(1)$}   
    $j \gets 0$\com*{$\Theta(1)$}   
    \While(\com*[f]{$\Theta(n)$} iteraciones){$j$ < Cardinal(Computadoras($res$.red))}{
      \While(\com*[f]{$\Theta(n)$} iteraciones){$k$ < Cardinal(Computadoras($res$.red))}{
        \If(\com*[f]{$\Theta(n \times I)$}){conectadas?($res$.red, $res$.IPsCompusPorID[j], $res$.IPsCompusPorID[k])}{
          \tipo{itConj} $it_{2} \gets$ crearIt(caminosMinimos($res$.red, $res$.IPsCompusPorID[j], $res$.IPsCompusPorID[k]))\;
          $res$.siguientesCompus[$j$][$k$] $\gets$ prim(fin(siguiente($it_{2}$)))\com*{$\Theta(1)$}   
        }
        $k \gets k + 1$\com*{$\Theta(1)$}  
      }
      $j \gets j + 1$\com*{$\Theta(1)$}  
    }
  \end{algoritmo}
 \datosAlgoritmo{} % Descripción
  {} % Pre
  {} % Post
  {$\Theta(n \times I + 4 n + n \times (n + I^2 + L) + n \times (n \times (n + I + n^2 \times n!)))$ = $\Theta(n \times I + n + n^2+ n \times I^2 + n \times L + n^3 + n^2 \times I + n^2 \times n!)$ = $\Theta(n^3 \times I^2 \times L + n^2 \times n!)$} % Complejidad
  {} % Justificacíón

  \begin{algoritmo}{iCrearPaquete}{\Inout{d}{DCNet},\In{p}{paquete}}{}
    % Crear e inicializar variables:
    \tipo{nat} $o \gets$ Obtener($p$.origen, $d$.IDsCompusPorIP)\com*{$\Theta(L)$}
    \tipo{nat} $dest \gets$ Obtener($p$.destino, $d$.IDsCompusPorIP)\com*{$\Theta(L)$}
    $it \gets$ CrearIt(($d$.paquetesEnEspera[$o$]).enConjunto)\com*{$\Theta(1)$}
    $it \gets$ Agregar(($d$.paquetesEnEspeta[$o$]).enConjunto, $p$)\com*{$\Theta(k)$}
    Definir($d$.paquetesEnEspera[$o$].porID, $p$.ID, $\langle$ $it$, $o$, $dest$ $\rangle$)\com*{$\Theta(log(k))$}
    encolar($d$.paquetesEnEspera[$o$].porPrioridad, $it$)\com*{$\Theta(log(k))$}
  \end{algoritmo}
  \datosAlgoritmo{} % Descripción
  {} % Pre
  {} % Post
  {$\Theta(L+log(k))$} % Complejidad
  {} % Justificacíón

  \begin{algoritmo}{iRed}{\In{d}{DCNet}}{Red}
    $res \gets$ $d$.Red \com*{$\Theta(1)$}
  \end{algoritmo}
  \datosAlgoritmo{} % Descripción
  {} % Pre
  {} % Post
  {$\Theta(1)$} % Complejidad
  {} % Justificacíón

  \begin{algoritmo}{iAvanzarSegundo}{\Inout{d}{DCNet}}{}
    \tipo{nat} $j \gets 0$\com*{$\Theta(1)$}
    \tipo{nat} $o$\com*{$\Theta(1)$}
    \tipo{nat} $dest$\com*{$\Theta(1)$}
    \tipo{paquete} $paq$\com*{$\Theta(1)$}
    \tipo{ItConj(paquete)} $i$ \com*{$\Theta(1)$}
    \While(\com*[f]{$\Theta(n)$}){$j$ < Cardinal(Computadoras($d$.red))}{
      \If(\com*[f]{$\Theta(1)$}){!(EsVacio?($d$.paquetesEnEspera[$j$]).enConjunto)}{
        $paq \gets$ Siguiente(desencolar(($d$.paquetesEnEspera[$j$]).porPrioridad))\com*{$\Theta(log(k))$}
        $o \gets$ (Obtener(($d$.paquetesEnEspera[$j$]).porID, $paq$.ID)).codOrigen\com*{$\Theta(log(k))$}
        $dest \gets$ (Obtener(($d$.paquetesEnEspera[$j$]).porID, $paq$.ID)).codDestino\com*{$\Theta(log(k))$}
        $i \gets$ (Obtener(($d$.paquetesEnEspera[$j$]).porID, $paq$.ID)).iPaquete\com*{$\Theta(log(k))$}
        Borrar(($d$.paquetesEnEspera[$j$]).porID, $paq$.ID)\com*{$\Theta(log(k))$}
        EliminarSiguiente(i)\com*{$\Theta(1)$}
        $d$.\#paqEnviados[$j$]++\com*{$\Theta(1)$}
        \If(\com*[f]{$\Theta(1)$}){!($d$.siquienteCompu[$j$][$dest$] = $dest$)}{
          \tipo{itConj(paquete)} $it \gets$ crearIt(($d$.paquetesEnEspera[$d$.siguienteCompu[$j$][$dest$]]).enConjunto)\com*{$\Theta(1)$}
          $it \gets$ AgregarRapido(($d$.paquetesEnEspera[$d$.siguienteCompu[$j$][$dest$]]).enConjunto, $p$)\com*{$\Theta(I)$ lo tomamos como $\Theta(1)$}
          Definir($d$.paquetesEnEspera[$d$.siguienteCompu[$j$][$dest$]].porID, $p$.ID, $\langle$ $it$, $d$.siguienteCompu[$j$][$dest$], $dest$, $\rangle$)\com*{$\Theta(log(k))$}
          encolar($d$.paquetesEnEspera[$d$.siguienteCompu[$j$][$dest$]].porPrioridad, $it$)\com*{$\Theta(log(k))$}
        }
      }
    }
    \tipo{nat} $k \gets 0$\com*{$\Theta(1)$}
    \tipo{nat} $h \gets 0$\com*{$\Theta(1)$}
      \While(\com*[f]{$\Theta(n)$ iteraciones}){k<Cardinal(Computadoras($d$.red))}{
        \If(\com*[f]{$\Theta(1)$}){$d$.\#paqEnviados[$k$]>d.\#paqEnviados[h]}{
          $h \gets k$\com*{$\Theta(1)$}
          k++\com*{$\Theta(1)$}
        }
    }
    $d$.laQueM\'asEnvi\'o $\gets$ $d$.IPsCompusPorID[h]\com*{$\Theta(1)$}
  \end{algoritmo}
  \datosAlgoritmo{} % Descripción
  {} % Pre
  {} % Post
  {$\Theta(n \times log(k))$ esta incluido en $\Theta(n \times (L+log(k)+log(n))$} % Complejidad
  {} % Justificacíón


  \begin{algoritmo}{iCaminoRecorrido}{\In{d}{DCNet}, \In{p}{paquete}}{lista(compu)}
    \tipo{nat} $j \gets 0$\com*{$\Theta(1)$}
    $res \gets$ Vacia()\com*{$\Theta(1)$}
    \While(\com*[f]{$\Theta(log(k) \times n)$}){!(iDefinido?(($d$.paquetesEnEspera[$j$]).porID), $p$.ID)}{
      $j$++\com*{$\Theta(1)$}
    }
    \tipo{nat} $o$\com*{$\Theta(1)$}
    $o \gets$ (Obtener(($d$.paquetesEnEspera[$j$]).porID, $p$.ID)).codOrigen\com*{$\Theta(log(k))$}
    \tipo{nat} $dest$\com*{$\Theta(1)$}
    $dest \gets$ (Obtener(($d$.paquetesEnEspera[$o$]).porID, $p$.ID)).codDestino\com*{$\Theta(log(k)$}
    \While(\com*[f]{$\Theta(n \times log(k))$}){!(Definido?(($d$.paquetesEnEspera[$o$]).porID), $p$.ID)}{
      AgregarAtras($res$, $d$.IPsCompusPorID[o])\com*{$\Theta(I)$ lo tomamos como $\Theta(1)$}
      $o \gets$ $d$.siguientesCompus[o][dest]\com*{$\Theta(1)$}
    }
    AgregarAtras($res$, $d$.IPsCompusPorID[$o$])\com*{$\Theta(I)$ lo tomamos como $\Theta(1)$}
  \end{algoritmo} 
  \datosAlgoritmo{} % Descripción
  {} % Pre
  {} % Post
  {$\Theta(n \times log(k))$} % Complejidad
  {} % Justificacíón

  \begin{algoritmo}{iCantidadEnviados}{\In{d}{DCNet}, \In{c}{compu}}{nat}
    \tipo{nat} $i \gets$ Obtener($d$.IDsCompusPorIP, $c$.IP)\com*{$\Theta(L)$}
    $res \gets$ $d$.\#paqEnviados[$i$]\com*{$\Theta(1)$}
  \end{algoritmo}
 \datosAlgoritmo{} % Descripción
  {} % Pre
  {} % Post
  {$\Theta(L)$} % Complejidad
  {} % Justificacíón

  \begin{algoritmo}{iEnEspera}{\In{d}{DCNet}, \In{c}{compu}}{conj(Paquete)}
    \tipo{nat} $i \gets$ Obtener($d$.IDsCompusPorIP, $c$.IP)\com*{$\Theta(L)$}
    $res \gets$ ($d$.paquetesEnEspera[$i$]).enConjunto\com*{$\Theta(1)$}
  \end{algoritmo}
 \datosAlgoritmo{} % Descripción
  {} % Pre
  {} % Post
  {$\Theta(L)$} % Complejidad
  {} % Justificacíón

  \begin{algoritmo}{iLaQueM\'{a}sEnvio}{\In{d}{DCNet}}{compu}
    $res \gets$ $d$.laQueM\'{a}sEnvio \com*{$\Theta(1)$}
  \end{algoritmo}
  \datosAlgoritmo{} % Descripción
  {} % Pre
  {} % Post
  {$\Theta(1)$} % Complejidad
  {Se pasa la computadora por referencia} % Justificacíón

  \begin{algoritmo}{paqueteEntr\'{a}nsito?}{\In{d}{DCNet}, \In{p}{paquete}}{bool}
    $res \gets$ false \com*{$\Theta(1)$}
    \tipo{nat} $i \gets 0$\com*{$\Theta(1)$}
    \While(\com*[f]{$\Theta(n)$}){$i$<longitud($d$.paquetesEnEspera)}{
      \If(\com*[f]{$\Theta(log(k))$}){definido?($p$, ($d$.paquetesEnEspera[$i$]).porID)}{
          $res \gets$ true\com*{$\Theta(1)$}
      }
    }
  \end{algoritmo}
  \datosAlgoritmo{} % Descripción
  {} % Pre
  {} % Post
  {$\Theta(n \times log(k))$} % Complejidad
  {} % Justificacíón

\end{Algoritmos}

