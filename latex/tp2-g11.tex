\documentclass[10pt, a4paper]{article}
\usepackage[paper=a4paper, left=1.5cm, right=1.5cm, bottom=1.5cm, top=3.5cm]{geometry}
\usepackage[latin1]{inputenc}
\usepackage[T1]{fontenc}
\usepackage[spanish,activeacute]{babel}
\usepackage{indentfirst}
\usepackage{fancyhdr}
\usepackage{latexsym}
\usepackage{lastpage}
\usepackage{aed2-symb,aed2-itef,aed2-tad,caratula}
\usepackage[colorlinks=true, linkcolor=blue]{hyperref}
\usepackage{calc}

\newcommand{\f}[1]{\text{#1}}
\newcommand{\fAux}{$_{\text{aux}}$}
\renewcommand{\paratodo}[2]{\ensuremath{\forall~#2: \text{#1}}}

% Macros de diseño provistas por la cátedra %

\usepackage{xspace}
\usepackage{xargs}
\usepackage{ifthen}

\newcommand{\moduloNombre}[1]{\textbf{#1}}

\let\NombreFuncion=\textsc
\let\TipoVariable=\texttt
\let\tipo=\texttt
\let\ModificadorArgumento=\textbf
\newcommand{\res}{$res$\xspace}
\newcommand{\tab}{\hspace*{7mm}}

\newcommandx{\TipoFuncion}[3]{%
  \NombreFuncion{#1}(#2) \ifx#3\empty\else $\to$ \res\,: \TipoVariable{#3}\fi%
}
\newcommand{\In}[2]{\ModificadorArgumento{in} \ensuremath{#1}\,: \TipoVariable{#2}\xspace}
\newcommand{\Out}[2]{\ModificadorArgumento{out} \ensuremath{#1}\,: \TipoVariable{#2}\xspace}
\newcommand{\Inout}[2]{\ModificadorArgumento{in/out} \ensuremath{#1}\,: \TipoVariable{#2}\xspace}
\newcommand{\Aplicar}[2]{\NombreFuncion{#1}(#2)}

\newlength{\IntFuncionLengthA}
\newlength{\IntFuncionLengthB}
\newlength{\IntFuncionLengthC}
%InterfazFuncion(nombre, argumentos, valor retorno, precondicion, postcondicion, complejidad, descripcion, aliasing)
\newcommandx{\InterfazFuncion}[9][4=true,6,7,8,9]{%
  \hangindent=\parindent
  \TipoFuncion{#1}{#2}{#3}\\%
  \textbf{Pre} $\equiv$ \{#4\}\\%
  \textbf{Post} $\equiv$ \{#5\}%
  \ifx#6\empty\else\\\textbf{Complejidad:} #6\fi%
  \ifx#7\empty\else\\\textbf{Descripci\'on:} #7\fi%
  \ifx#8\empty\else\\\textbf{Aliasing:} #8\fi%
  \ifx#9\empty\else\\\textbf{Requiere:} #9\fi%
}

\newenvironment{Interfaz}{%
  \parskip=2ex%
  \noindent\textbf{\Large Interfaz}%
  \par%
}{}

\newenvironment{Representacion}{%
  \vspace*{2ex}%
  \noindent\textbf{\Large Representaci\'on}%
  \vspace*{2ex}%
}{}

\newenvironment{Algoritmos}{%
  \vspace*{2ex}%
  \noindent\textbf{\Large Algoritmos}%
  \vspace*{2ex}%
}{}


\newcommand{\Encabezado}[1]{
  \vspace*{1ex}\par\noindent\textbf{\large #1}\par
}

\newenvironmentx{Estructura}[2][2={estr}]{%
  \par\vspace*{2ex}%
  \TipoVariable{#1} \textbf{se representa con} \TipoVariable{#2}%
  \par\vspace*{1ex}%
}{%
  \par\vspace*{2ex}%
}%

\newboolean{EstructuraHayItems}
\newlength{\lenTupla}
\newenvironmentx{Tupla}[1][1={estr}]{%
    \settowidth{\lenTupla}{\hspace*{3mm}donde \TipoVariable{#1} es \TipoVariable{tupla}$($}%
    \addtolength{\lenTupla}{\parindent}%
    \hspace*{3mm}donde \TipoVariable{#1} es \TipoVariable{tupla}$($%
    \begin{minipage}[t]{\linewidth-\lenTupla}%
    \setboolean{EstructuraHayItems}{false}%
}{%
    $)$%
    \end{minipage}
}

\newcommandx{\tupItem}[3][1={\ }]{%
    %\hspace*{3mm}%
    \ifthenelse{\boolean{EstructuraHayItems}}{%
        ,#1%
    }{}%
    \emph{#2}: \TipoVariable{#3}%
    \setboolean{EstructuraHayItems}{true}%
}

\newcommandx{\tupItemNL}[3][1={\ }]{%
    %\hspace*{3mm}%
    \ifthenelse{\boolean{EstructuraHayItems}}{%
        ,\\#1%
    }{}%
    \emph{#2}: \TipoVariable{#3}%
    \setboolean{EstructuraHayItems}{true}%
}

\newcommandx{\RepFc}[3][1={estr},2={e}]{%
  \tadOperacion{Rep}{#1}{bool}{}%
  \tadAxioma{Rep($#2$)}{#3}%
}%

\newcommandx{\Rep}[3][1={estr},2={e}]{%
  \tadOperacion{Rep}{#1}{bool}{}%
  \tadAxioma{Rep($#2$)}{true \ssi #3}%
}%

\newcommandx{\Abs}[5][1={estr},3={e}]{%
  \tadOperacion{Abs}{#1/#3}{#2}{Rep($#3$)}%
  \settominwidth{\hangindent}{Abs($#3$) \igobs #4: #2 $\mid$ }%
  \addtolength{\hangindent}{\parindent}%
  Abs($#3$) \igobs #4: #2 $\mid$ #5%
}%

\newcommandx{\AbsFc}[4][1={estr},3={e}]{%
  \tadOperacion{Abs}{#1/#3}{#2}{Rep($#3$)}%
  \tadAxioma{Abs($#3$)}{#4}%
}%

\newcommand{\DRef}{\ensuremath{\rightarrow}}

% Macros de diseño propias %

\usepackage{scrextend} % Para poder indentar bloques

\newenvironment{paramFormales}{
  \textbf{par\'ametros formales}
  \vspace{-0.5em}
  \list{}{\leftmargin8em \topsep0.2em \itemsep0.25em \labelsep2em}
}{
  \endlist 
}

\newcommand{\paramGeneros}[1]{\item[\textbf{g\'eneros}] #1}

\newcommand{\paramFuncion}[1]{\item[\textbf{funci\'on}] \parbox[t]{\textwidth-2\parindent-1.7cm}{#1}}

\newcommand{\seExplicaCon}[1]{\parbox{3cm}{\textbf{se explica con}:} \tadNombre{#1}}

\newcommand{\generos}[1]{\parbox{3cm}{\textbf{g\'eneros}:} #1}

\usepackage[noresetcount]{algorithm2e}
\usepackage{float}

\NoCaptionOfAlgo\SetAlgoLongEnd\LinesNumbered\SetAlgoLined\RestyleAlgo{ruled}\IncMargin{1em}\DontPrintSemicolon\SetArgSty{}\SetCommentSty{textsf}\SetFuncSty{textsf}

\newenvironment{algoritmo}[3]{
  \setcounter{AlgoLine}{0}
  \begin{algorithm}[H]
  \caption{\TipoFuncion{#1}{#2}{#3}}
}{
  \end{algorithm}
  \vspace{0em}
}

\newenvironment{contAlgoritmo}[1]{
  \begin{algorithm}[H]
  \caption{\NombreFuncion{#1} \emph{(cont.)}}
}{
  \end{algorithm}
}

\newcommand{\complejidad}[1]{\textbf{Complejidad:} #1 \vspace{1em}}

\SetKwComment{com}{ $\triangleright$ }{}
\SetKwFunction{delete}{delete}
\def\NULL{\textrm{NULL}}
\def\new{\textbf{\&}}

\sloppy

\hypersetup{%
 % Para que el PDF se abra a pagina completa.
 pdfstartview= {FitH \hypercalcbp{\paperheight-\topmargin-1in-\headheight}},
 pdfauthor={Grupo 11},
 pdfkeywords={HOLA HOLA},
 pdfsubject={Trabajo pr\'actico 2 - Dise\'no - DCNet}
}

\parskip=5pt % 10pt es el tamaño de fuente

% Pongo en 0 la distancia extra entre ítemes.
\let\olditemize\itemize
\def\itemize{\olditemize\itemsep=0pt}

% Acomodo fancyhdr.
\pagestyle{fancy}
\thispagestyle{fancy}
\addtolength{\headheight}{1pt}
\lhead{Algoritmos y Estructuras de Datos II}
\rhead{Trabajo Pr\'actico 2 - Dise\~no - DCNet}
\cfoot{\thepage /\pageref{LastPage}}
\renewcommand{\footrulewidth}{0.4pt}

\author{Grupo 11}
\date{}
\title{Trabajo pr\'actico 2 - Dise\~no - DCNet}

\def\Materia{Algoritmos y Estructuras de Datos II}
\def\Titulo{{\small De los creadores de sacarCompu...} \\ \vspace*{1.5ex} Trabajo pr\'actico 2}
\def\Subtitulo{Dise\~no - DCNet}
\def\Grupo{Grupo 11}
\integrante{Frizzo, Franco}{013/14}{francofrizzo@gmail.com}
\integrante{Mart\'inez, Manuela}{160/14}{martinez.manuela.22@gmail.com}
\integrante{Rabinowicz, Luc\'ia}{105/14}{lu.rabinowicz@gmail.com}
\integrante{Weber, Andr\'es}{923/13}{herr.andyweber@gmail.com}

\begin{document}

\maketitle
\newpage\null\thispagestyle{empty}\newpage

\newcommand{\avlKS}{diccAVL(\ensuremath{\kappa}, \ensuremath{\sigma})} % Macro para diccAVL(k, s)

\section{M\'odulo Diccionario AVL}

\begin{Interfaz}
  
  \begin{paramFormales}
    \paramGeneros{$\kappa$, $\sigma$}

    \paramFuncion{
      \InterfazFuncion{$\bullet = \bullet$}{\In{k_1}{$\kappa$}, \In{k_2}{$\kappa$}}{\tipo{bool}}
      {$res \igobs (k_1 = k_2)$}
      [$\Theta(equal(k_1, k_2))$]
      [funci\'on de igualdad de $\kappa$]
    }

    \paramFuncion{
      \InterfazFuncion{$\bullet \leq \bullet$}{\In{k_1}{$\kappa$}, \In{k_2}{$\kappa$}}{\tipo{bool}}
      {$res \igobs (k_1 \leq k_2)$}
      [$\Theta(order(k_1, k_2))$]
      [funci\'on de comparaci\'on por orden total estricto de $\kappa$]
    }

    \paramFuncion{
      \InterfazFuncion{Copiar}{\In{k}{$\kappa$}}{$\kappa$}
      {$res \igobs k$}
      [$\Theta(copy(k))$]
      [funci\'on de copia de $\kappa$]
    }

    \paramFuncion{
      \InterfazFuncion{Copiar}{\In{s}{$\sigma$}}{$\sigma$}
      {$res \igobs s$}
      [$\Theta(copy(s))$]
      [funci\'on de copia de $\sigma$]
    }
  \end{paramFormales}

  \seExplicaCon{Diccionario($\kappa$, $\sigma$)}

  \generos{\tipo{\avlKS}}

  \Encabezado{Operaciones de diccionario}

    \InterfazFuncion{Vacio}{}{\avlKS}
    {$res \igobs$ vac\'io}
    [$\Theta(1)$]
    [Crea y devuelve un diccionario AVL vac\'io.]

    \InterfazFuncion{Definir}{\Inout{d}{\avlKS}, \In{k}{$\kappa$}, \In{s}{$\sigma$}}{}
    [$d \igobs d_0$]
    {$d \igobs$ definir($k$, $s$, $d_0$)}
    [$\Theta(\log(n) \times order(k) + copy(k) + copy(s))$]
    [Define en el diccionario la clave pasada por par\'ametro con el significado pasado por par\'ametro. En caso de que la clave ya est\'e definida, sobreescribe su significado con el nuevo.]

    \InterfazFuncion{Borrar}{\Inout{d}{\avlKS}, \In{k}{$\kappa$}}{}
    [$d \igobs d_0 \land$ def?($k$, $d$)]
    {$d \igobs$ borrar($k$, $d_0$)}
    [$\Theta(\log(n) \times order(k))$]
    [Elimina del diccionario la clave pasada por par\'ametro.]

    \InterfazFuncion{\#Claves}{\In{d}{\avlKS}}{nat}
    {$res \igobs$ \#(claves($d$))}
    [$\Theta(1)$]
    [Devuelve la cantidad de claves del diccionario.]

    \InterfazFuncion{Definido?}{\In{d}{\avlKS}, \In{k}{$\kappa$}}{bool}
    {$res \igobs$ def?($k$, $d$)}
    [$\Theta(\log(n) \times order(k))$]
    [Devuelve true si y solo si la clave pasada por par\'ametro est\'a definida en el diccionario.]

    \InterfazFuncion{Obtener}{\In{d}{\avlKS}, \In{k}{$\kappa$}}{$\sigma$}
    [def?($k$, $d$)]
    {$res \igobs$ obtener($k$, $d$)}
    [$\Theta(\log(n) \times order(k))$]
    [Devuelve el significado con el que la clave pasada por par\'ametro est\'a definida en el diccionario.]
    [El significado se pasa por referencia. Modificarlo implica cambiarlo en la estructura interna del diccionario.]

\end{Interfaz}

\begin{Representacion}

  \begin{Estructura}{diccAVL($\kappa$, $\sigma$)}[estrAVL]

    \begin{Tupla}[estrAVL]
      \tupItemNL{raiz}{puntero(nodo)}%
      \tupItemNL{cantNodos}{nat}%
    \end{Tupla}

    \begin{Tupla}[nodo]
      \tupItemNL{clave}{$\kappa$}%
      \tupItemNL{significado}{$\sigma$}%
      \tupItemNL{padre}{puntero(nodo)}%
      \tupItemNL{izq}{puntero(nodo)}%
      \tupItemNL{der}{puntero(nodo)}%
      \tupItemNL{altSubarbol}{nat}%
    \end{Tupla}

  \end{Estructura} 

  \Rep[estrAVL][e]{
    $e.cantNodos$ = \#(nodos($e$)) $\land$ \\
    \#(claves($e$)) = \#(nodos($e$)) $\land$ \\
    $(\paratodo{nodo}{n})$ ($n$ $\in$ nodos($e$) $\impluego$ (\\
    $n.altSubarbol$ = altura(sub\'arbol($\&n$)) $\land$ \\
    m\'ax(altura(sub\'arbol($n.izq$)), altura(sub\'arbol($n.der$))) $-$ \\
    m\'in(altura(sub\'arbol($n.izq$)), altura(sub\'arbol($n.der$))) $\leq 1 \land$ \\
    $(\paratodo{nodo}{n'})$ (($n' \in$ nodos(sub\'arbol($n$))) $\implies$ ($n' \in$ nodos(sub\'arbol($n.der$)) $\Leftrightarrow$ $\neg (n'.clave \leq n.clave)$)) $\land$ \\
    $*(n.izq) \neq *(n.der)$ $\land$ $(n.padre = \NULL \Leftrightarrow \&n = e.raiz) \yluego$ \\
    $((\&n \neq e.raiz) \impluego (\paratodo{nodo}{n'})(n.padre = \&n' \Leftrightarrow n'.izq = \&n \lor n'.der = \&n)$)))
  }

  ~

  \Abs[estrAVL]{dicc($\kappa$, $\sigma$))}[e]{$d$}{
    ($\paratodo{k}{\kappa}$) ((def?($k$, $d$)) $\igobs$ ($k$ $\in$ claves($e$)) $\yluego$
    (def?($k$, $d$) $\impluego$ obtener($k$, $d$) $\igobs$ significado($k$, $e$)))
  }

  ~

  \tadOperacion{hijos}{nodo}{conj(nodo)}{}
  \tadAxioma{hijos($n$)}{\IF $n.izq$ = NULL THEN $\emptyset$ ELSE Ag($*$($n.izq$), hijos($*$($n.izq$))) FI \\
  $\cup$ \IF $n.der$ = NULL THEN $\emptyset$ ELSE Ag($*$($n.der$), hijos($*$($n.der$))) FI}

  ~

  \tadOperacion{nodos}{estrAVL}{conj(nodo)}{}
  \tadAxioma{nodos($e$)}{\IF $e.raiz$ = NULL THEN $\emptyset$ ELSE Ag($*$($e.raiz$), hijos($*$($e.raiz$))) FI}

  ~

  % \tadOperacion{sub\'arbol}{nodo}{estrAVL}{}
  % \tadAxioma{sub\'arbol($n$)}{$\langle \&n$, $1 + \#$(hijos($n$))$\rangle$}

  % ~

  \tadOperacion{sub\'arbol}{puntero(nodo)}{estrAVL}{}
  \tadAxioma{sub\'arbol($p$)}{$\langle p$, $1 + \#$(hijos($*$($p$)))$\rangle$}

  ~

  \tadOperacion{claves}{estrAVL}{conj($\kappa$)}{}
  \tadAxioma{claves($e$)}{\IF $e.raiz = \NULL$ THEN $\emptyset$ ELSE Ag($e.raiz \to clave$, claves(sub\'arbol($e.raiz \to izq$)) $\cup$ claves(sub\'arbol($e.raiz \to der$))) FI}

  ~

  \tadOperacion{altura}{estrAVL}{nat}{}
  \tadAxioma{altura($e$)}{\IF $e.raiz = \NULL$ THEN 0 ELSE 1 + m\'ax(altura(sub\'arbol($e.raiz \to izq$)), altura(sub\'arbol($e.raiz \to der$))) FI}

  ~

  \tadOperacion{significado}{estrAVL/e,$\kappa$/k}{$\sigma$}{$k$ $\in$ claves($e$)}
  \tadAxioma{significado($e$, $k$)}{\IF $e.raiz \to clave = k$ THEN $e.raiz \to significado$ ELSE {\IF $k$ $\in$ claves(sub\'arbol($e.raiz \to izq$)) THEN significado($k$, sub\'arbol($e.raiz \to izq$)) ELSE significado($k$, sub\'arbol($e.raiz \to der$)) FI} FI}

\end{Representacion}

\begin{Algoritmos}

  \SetKwFunction{fdb}{FDB}
  \SetKwFunction{raizq}{RotarAIzquierda}
  \SetKwFunction{rader}{RotarADerecha}
  \SetKwFunction{buscar}{Buscar}
  \SetKwFunction{max}{max}
  \SetKwFunction{rebalancear}{RebalancearArbol}
  \SetKwFunction{recalcAlt}{RecalcularAltura}
  % \SetKwFunction{recalcAlts}{RecalcularAlturas}

  \begin{algoritmo}{iVacio}{}{estrAVL}
    $res \gets \langle$NULL, 0$\rangle$ \com*{$\Theta(1)$}
  \end{algoritmo}
  \datosAlgoritmo{} % Descripción
  {} % Pre
  {} % Post
  {$\Theta(1)$} % Complejidad
  {} % Justificación

  \begin{algoritmo}{iDefinir}{\Inout{e}{estrAVL}, \In{k}{$\kappa$}, \In{s}{$\sigma$}}{}
    \tipo{puntero(nodo)} $padre \gets \NULL$ \com*{$\Theta(1)$}
    \tipo{puntero(nodo)} $lugar \gets$ \buscar{$e$, $k$, $padre$} \com*{$\Theta(\log(n) \times order(k))$}
    \eIf(\com*[f]{$\Theta(1)$}){$lugar \neq \NULL$}{
      $(lugar \to significado) \gets$ \copiar{$s$} \com*{$\Theta(copy(s))$}
    }{
      \tipo{puntero(nodo)} $nuevo \gets \new \langle$\copiar{$k$}, \copiar{$s$}, NULL, NULL, NULL, $1 \rangle$ \tcp{Reservamos memoria para el nuevo nodo} \com*{$\Theta(copy(k) + copy(s))$}
      \eIf(\com*[f]{$\Theta(order(k))$}){$k \leq (padre \to clave)$}{
        $(padre \to izq) \gets nuevo$ \com*{$\Theta(1)$}
      }{
        $(padre \to der) \gets nuevo$ \com*{$\Theta(1)$}
      }
      $(nuevo \to padre) \gets padre$ \com*{$\Theta(1)$}
      % \recalcAlts($padre$) \;
      \rebalancear($padre$) \com*{$\Theta(\log(n))$}
      $e.cantNodos ++$ \com*{$\Theta(1)$}
    }
  \end{algoritmo}
  \datosAlgoritmo{} % Descripción
  {} % Pre
  {} % Post
  {$\Theta(\log(n) \times order(k) + copy(k) + copy(s))$} % Complejidad
  {La funci\'on tiene llamadas a funciones con complejidad $\Theta(\log(n) \times order(k)$ y $\Theta(copy(k) + copy(s))$.} % Justificacíón

  \begin{algoritmo}{iObtener}{\In{e}{estrAVL}, \In{k}{$\kappa$}}{$\sigma$}
    \tipo{puntero(nodo)} $padre \gets \NULL$ \com*{$\Theta(1)$}
    \tipo{puntero(nodo)} $lugar \gets$ \buscar{$e$, $k$, $padre$} \com*{$\Theta(\log(n) \times order(k))$}
    $res \gets (lugar \to significado)$ \com*{$\Theta(1)$}
  \end{algoritmo}
  \datosAlgoritmo{} % Descripción
  {} % Pre
  {} % Post
  {$\Theta(\log(n) \times order(k))$} % Complejidad
  {La funci\'on tiene llamadas a funciones con complejidad $\Theta(\log(n) \times order(k)$ y $\Theta(copy(k) + copy(s))$.} % Justificacíón

  \begin{algoritmo}{\#Claves}{\In{e}{estrAVL}}{nat}
    $res \gets e.cantNodos$ \com*{$\Theta(1)$}
  \end{algoritmo}
  \datosAlgoritmo{} % Descripción
  {} % Pre
  {} % Post
  {$\Theta(1)$} % Complejidad
  {} % Justificación

  \begin{algoritmo}{Definido?}{\In{e}{estrAVL}, \In{k}{$\kappa$}}{bool}
    \tipo{puntero(nodo)} $padre \gets \NULL$ \com*{$\Theta(1)$}
    \tipo{puntero(nodo)} $lugar \gets$ \buscar{$e$, $k$, $padre$} \com*{$\Theta(\log(n) \times order(k))$}
    $res \gets (lugar \neq \NULL)$ \com*{$\Theta(1)$}
  \end{algoritmo}
  \datosAlgoritmo{} % Descripción
  {} % Pre
  {} % Post
  {$\Theta(\log(n) \times order(k))$} % Complejidad
  {} % Justificación

  \begin{algoritmo}{iBorrar}{\Inout{e}{estrAVL}, \In{k}{$\kappa$}}{}
    \tipo{puntero(nodo)} $padre \gets \NULL$ \com*{$\Theta(1)$}
    \tipo{puntero(nodo)} $lugar \gets$ \buscar{$e$, $k$, $padre$} \com*{$\Theta(\log(n) \times order(k))$}
    \uIf(\com*[f]{$\Theta(1)$}){$lugar \to izq = \NULL \land lugar \to der = \NULL$}{
      \eIf(\com*[f]{$\Theta(1)$}){$padre \neq \NULL$}{
        \eIf(\com*[f]{$\Theta(1)$}){$padre \to izq = lugar$}{
          $(padre \to izq) \gets \NULL$ \com*{$\Theta(1)$}
        }{
          $(padre \to der) \gets \NULL$ \com*{$\Theta(1)$}
        }
        % \recalcAlts{$padre$} \;
        \rebalancear{$padre$} \com*{$\Theta(\log(n))$}
      }{
        $e.raiz = \NULL$ \com*{$\Theta(1)$}
      }
    }\uElseIf(\com*[f]{$\Theta(1)$}){$lugar \to der = \NULL$}{
      $(lugar \to izq \to padre) \gets padre$ \com*{$\Theta(1)$}
      \eIf(\com*[f]{$\Theta(1)$}){$padre \neq \NULL$}{
        \eIf(\com*[f]{$\Theta(1)$}){$padre \to izq = lugar$}{
          $(padre \to izq) \gets (lugar \to izq)$ \com*{$\Theta(1)$}
        }{
          $(padre \to der) \gets (lugar \to izq)$ \com*{$\Theta(1)$}
        }
        % \recalcAlts{$padre$} \;
        \rebalancear{$padre$} \com*{$\Theta(\log(n))$}
      }{
        $e.raiz \gets lugar \to izq$ \com*{$\Theta(1)$}
      }
    }\uElseIf(\com*[f]{$\Theta(1)$}){$lugar \to izq = \NULL$}{
      $(lugar \to der \to padre) \gets padre$ \com*{$\Theta(1)$}
      \eIf(\com*[f]{$\Theta(1)$}){$padre \neq \NULL$}{
        \eIf(\com*[f]{$\Theta(1)$}){$padre \to izq = lugar$}{
          $(padre \to izq) \gets (lugar \to der)$ \com*{$\Theta(1)$}
        }{
          $(padre \to der) \gets (lugar \to der)$ \com*{$\Theta(1)$}
        }
        % \recalcAlts{$padre$} \;
        \rebalancear{$padre$} \com*{$\Theta(\log(n))$}
      }{
        $e.raiz \gets lugar \to izq$ \com*{$\Theta(1)$}
      }
    }
  \end{algoritmo}

  \begin{contAlgoritmo}{iBorrar}
    \Else{
      \tipo{puntero(nodo)} $reemplazo \gets (lugar \to der)$ \com*{$\Theta(1)$}
      \eIf(\com*[f]{$\Theta(1)$}){$(reemplazo \to izq = \NULL)$}{
        \eIf(\com*[f]{$\Theta(1)$}){$padre \neq \NULL$}{
          \eIf(\com*[f]{$\Theta(1)$}){$(padre \to izq) = lugar$}{
            $(padre \to izq) \gets reemplazo$ \com*{$\Theta(1)$}
          }{
            $(padre \to der) \gets reemplazo$ \com*{$\Theta(1)$}
          }
        }{
          $e.raiz \gets reemplazo$ \com*{$\Theta(1)$}
        }
        $(reemplazo \to padre) \gets padre$ \com*{$\Theta(1)$}
        $(reemplazo \to izq) \gets lugar \to izq$ \com*{$\Theta(1)$}
        $(lugar \to izq \to padre) \gets reemplazo$ \com*{$\Theta(1)$}
        % \recalcAlts{$reemplazo$} \;
        \rebalancear{$reemplazo$} \com*{$\Theta(\log(n))$}
      }{
        \While(\com*[f]{$\Theta(\log(n))$ iteraciones}){$(reemplazo \to izq) \neq \NULL$}{
          $reemplazo \gets (reemplazo \to izq)$ \com*{$\Theta(1)$}
        }
        \tipo{puntero(nodo)} $padreReemplazo \gets (reemplazo \to padre)$ \com*{$\Theta(1)$}
        \eIf(\com*[f]{$\Theta(1)$}){$padre \neq \NULL$}{
          \eIf(\com*[f]{$\Theta(1)$}){$padre \to izq = lugar$}{
            $(padre \to izq) \gets reemplazo$ \com*{$\Theta(1)$}
          }{
            $(padre \to der) \gets reemplazo$ \com*{$\Theta(1)$}
          }
        }{
          $e.raiz \gets reemplazo$ \com*{$\Theta(1)$}
        }
        $(reemplazo \to padre) \gets padre$ \com*{$\Theta(1)$}
        $(reemplazo \to izq) \gets lugar \to izq$ \com*{$\Theta(1)$}
        $(lugar \to izq \to padre) \gets reemplazo$ \com*{$\Theta(1)$}
        $(padreReemplazo \to izq) \gets (reemplazo \to der)$ \com*{$\Theta(1)$}
        \If(\com*[f]{$\Theta(1)$}){$(reemplazo \to der) \neq \NULL$}{
          $(reemplazo \to der \to padre) \gets padreReemplazo$ \com*{$\Theta(1)$}
        }
        $(reemplazo \to der) \gets (lugar \to der)$ \com*{$\Theta(1)$}
        $(lugar \to der \to padre) \gets reemplazo$ \com*{$\Theta(1)$}
        % \recalcAlts{$reemplazo$} \;
        \rebalancear{$reemplazo$} \com*{$\Theta(\log(n))$}
      }
    }
    \delete{$lugar$} \com*{$\Theta(1)$}
  \end{contAlgoritmo}
  \datosAlgoritmo{} % Descripción
  {} % Pre
  {} % Post
  {$\Theta(\log(n) \times order(k))$} % Complejidad
  {} % Justificación

  \begin{algoritmo}{iBuscar}{\In{e}{estrAVL}, \In{k}{$\kappa$}, \Out{padre}{puntero(nodo)}}{puntero(nodo)}
    $padre \gets \NULL$  \com*{$\Theta(1)$}
    $actual \gets e.raiz$  \com*{$\Theta(1)$}
    \While(\com*[f]{$\Theta(\log(n))$ iteraciones}){$actual \neq \NULL \yluego (actual \to clave \neq k)$}{ % ¡Preguntar por el y luego!
      $padre \gets actual$ \com*{$\Theta(1)$}
      \eIf(\com*[f]{$\Theta(order(k))$}){$k \leq (padre \to clave)$}{
        $actual \gets (actual \to izq)$ \com*{$\Theta(1)$}
      }{
        $actual \gets (actual \to der)$ \com*{$\Theta(1)$}
      }
    }
    $res \gets actual$ \com*{$\Theta(1)$}
  \end{algoritmo}
  \datosAlgoritmo{} % Descripción
  {} % Pre
  {} % Post
  {$\Theta(\log(n) \times order(k))$} % Complejidad
  {} % Justificación

  % \begin{algoritmo}{iRecalcularAlturas}{\In{n}{puntero(nodo)}}{}
  %   \tipo{puntero(nodo)} $p \gets n$ \com*{$\Theta(1)$}
  %   \While(\com*[f]{$\Theta(\log(n))$ iteraciones}){$p \neq \NULL$}{
  %     \uIf{$p \to izq \neq \NULL \land p \to der \neq \NULL$}{
  %       $(p \to altSubarbol) \gets 1$ + \max{$p \to izq \to altSubarbol$, $p \to der \to altSubarbol$} \com*{$\Theta(1)$}
  %     }\uElseIf{$p \to izq \neq \NULL$}{
  %       $(p \to altSubarbol) \gets 1 + (p \to izq \to altSubarbol)$ \com*{$\Theta(1)$}
  %     }\uElseIf{$p \to der \neq \NULL$}{
  %       $(p \to altSubarbol) \gets 1 + (p \to der \to altSubarbol)$ \com*{$\Theta(1)$}
  %     }\Else{
  %       $(p \to altSubarbol) \gets 1$ \com*{$\Theta(1)$}
  %     }
  %     $p \gets (p \to padre)$ \com*{$\Theta(1)$}
  %   }
  % \end{algoritmo}
  % \datosAlgoritmo{} % Descripción
  % {} % Pre
  % {} % Post
  % {$\Theta(\log(n))$} % Complejidad
  % {} % Justificación

  \begin{algoritmo}{iRecalcularAltura}{\In{n}{puntero(nodo)}}{}
    \uIf(\com*[f]{$\Theta(1)$}){$n \to izq \neq \NULL \land n \to der \neq \NULL$}{
      $(n \to altSubarbol) \gets 1$ + \max{$n \to izq \to altSubarbol$, $n \to der \to altSubarbol$} \com*{$\Theta(1)$}
    }\uElseIf(\com*[f]{$\Theta(1)$}){$n \to izq \neq \NULL$}{
      $(n \to altSubarbol) \gets 1 + (n \to izq \to altSubarbol)$ \com*{$\Theta(1)$}
    }\uElseIf(\com*[f]{$\Theta(1)$}){$n \to der \neq \NULL$}{
      $(n \to altSubarbol) \gets 1 + (n \to der \to altSubarbol)$ \com*{$\Theta(1)$}
    }\Else{
      $(n \to altSubarbol) \gets 1$ \com*{$\Theta(1)$}
    }
  \end{algoritmo}
  \datosAlgoritmo{} % Descripción
  {} % Pre
  {} % Post
  {$\Theta(1)$} % Complejidad
  {} % Justificación

  \begin{algoritmo}{iFBD}{\In{n}{puntero(nodo)}}{int}
    \tipo{int} $altIzq \gets n \to izq = \NULL$ ? 0 : $n \to izq \to altSubarbol$ \com*{$\Theta(1)$}
    \tipo{int} $altDer \gets n \to der = \NULL$ ? 0 : $n \to izq \to altSubarbol$ \com*{$\Theta(1)$}
    $res \gets altDer - altIzq$ \com*{$\Theta(1)$}
  \end{algoritmo}
  \datosAlgoritmo{} % Descripción
  {} % Pre
  {} % Post
  {$\Theta(1)$} % Complejidad
  {} % Justificación

  \begin{algoritmo}{iRotarAIzquierda}{\In{n}{puntero(nodo)}}{}
    \If(\com*[f]{$\Theta(1)$}){$n.padre \neq \NULL$}{
      \eIf(\com*[f]{$\Theta(1)$}){$n.padre \to izq = n$}{
        $(n \to padre \to izq) \gets n \to der$ \com*{$\Theta(1)$}
      }{
        $(n \to padre \to der) \gets n \to der$ \com*{$\Theta(1)$}
      }
    }
    $(n \to der \to padre) \gets n \to padre$ \com*{$\Theta(1)$} 
    $n \to padre \gets n \to der$ \com*{$\Theta(1)$}
    $n \to der \gets (n \to der \to izq)$ \com*{$\Theta(1)$}
    \If(\com*[f]{$\Theta(1)$}){$n \to der \neq \NULL$}{
      $(n \to der \to padre) \gets n$ \com*{$\Theta(1)$}
    }
    $(n \to padre \to izq) \gets n$ \com*{$\Theta(1)$}
    \recalcAlt{$n$} \com*{$\Theta(1)$}
    \recalcAlt{$n \to padre$} \com*{$\Theta(1)$}
  \end{algoritmo}
  \datosAlgoritmo{} % Descripción
  {} % Pre
  {} % Post
  {$\Theta(1)$} % Complejidad
  {} % Justificación

  \begin{algoritmo}{iRotarADerecha}{\In{n}{puntero(nodo)}}{}
    \If(\com*[f]{$\Theta(1)$}){$n \to padre \neq \NULL$}{
      \eIf(\com*[f]{$\Theta(1)$}){$n \to padre \to izq = n$}{
        $(n \to padre \to izq) \gets n \to izq$ \com*{$\Theta(1)$}
      }{
        $(n \to padre \to der) \gets n \to izq$ \com*{$\Theta(1)$}
      }
    }
    $(n \to izq \to padre) \gets n \to padre$ \com*{$\Theta(1)$} 
    $n \to padre \gets n \to izq$ \com*{$\Theta(1)$}
    $n \to izq \gets (n \to izq \to der)$ \com*{$\Theta(1)$}
    \If(\com*[f]{$\Theta(1)$}){$n \to izq \neq \NULL$}{
      $(n \to izq \to padre) \gets n$ \com*{$\Theta(1)$}
    }
    $(n \to padre \to der) \gets n$ \com*{$\Theta(1)$}
    \recalcAlt{$n$} \com*{$\Theta(1)$}
    \recalcAlt{$n \to padre$} \com*{$\Theta(1)$}
  \end{algoritmo}
  \datosAlgoritmo{} % Descripción
  {} % Pre
  {} % Post
  {$\Theta(1)$} % Complejidad
  {} % Justificación

  \begin{algoritmo}{iRebalancearArbol}{\In{n}{puntero(nodo)}}{}
    \tipo{puntero(nodo)} $p \gets n$ \com*{$\Theta(1)$}
    \While(\com*[h]{$\Theta(\log(n))$ iteraciones}){$p \neq \NULL$}{ 
      \recalcAlt{$p$} \com*{$\Theta(1)$}
      \tipo{int} $fdb1 \gets$ \fdb{$p$} \com*{$\Theta(1)$}
      \uIf(\com*[f]{$\Theta(1)$}){$fdb1 = 2$}{
        \tipo{puntero(nodo)} $q \gets (p \to der)$ \com*{$\Theta(1)$}
        \tipo{int} $fdb2 \gets$ \fdb{$q$} \com*{$\Theta(1)$}
        \uIf(\com*[f]{$\Theta(1)$}){$fdb2 = 1 \lor fdb2 = 0$}{
          \raizq{$p$} \com*{$\Theta(1)$}
          $p \gets q$ \com*{$\Theta(1)$}
        }\ElseIf(\com*[f]{$\Theta(1)$}){$fbd2 = -1$}{
          \rader{$q$} \com*{$\Theta(1)$}
          \raizq{$p$} \com*{$\Theta(1)$}
          $p \gets (q \to padre)$ \com*{$\Theta(1)$}
        }
      }\ElseIf{$fdb1 = -2$}{
        \tipo{puntero(nodo)} $q \gets (p \to izq)$ \com*{$\Theta(1)$}
        \tipo{int} $fdb2 \gets$ \fdb{$q$} \com*{$\Theta(1)$}
        \uIf(\com*[f]{$\Theta(1)$}){$fdb2 = -1 \lor fdb2 = 0$}{
          \rader{$p$} \com*{$\Theta(1)$}
          $p \gets q$ \com*{$\Theta(1)$}
        }\ElseIf(\com*[f]{$\Theta(1)$}){$fbd2 = 1$}{
          \raizq{$q$} \com*{$\Theta(1)$}
          \rader{$p$} \com*{$\Theta(1)$}
          $p \gets (q \to padre)$ \com*{$\Theta(1)$}
        }
      }
      $p \gets (p \to padre)$ \com*{$\Theta(1)$}
    }
  \end{algoritmo}
  \datosAlgoritmo{} % Descripción
  {} % Pre
  {} % Post
  {$\Theta(\log(n))$} % Complejidad
  {} % Justificación

\end{Algoritmos}

\section{M\'{o}dulo Heap($\alpha$)}

\begin{Interfaz}
  
  \begin{paramFormales}
    \paramGeneros{$\alpha$}

%    \paramFuncion{
%      \InterfazFuncion{$\bullet = \bullet$}{\In{a_{1}}{$\alpha$}, \In{a_{2}}{$\alpha$}}{bool}
%      {$res \igobs (a_{1} = a_{2})$}
%      [$\Theta(equal(a_{1}, a_{2}))$]
%      [funci\'on de igualdad de $\alpha$]
%    }

    \paramFuncion{
      \InterfazFuncion{$\bullet < \bullet$}{\In{a_{1}}{$\alpha$}, \In{a_{a}}{$\alpha$}}{bool}
      {$res \igobs (a_{1} \leq a_{2})$}
      [$\Theta(compare(a_{1}, a_{2}))$]
      [funci\'{o}n de comparaci\'{o}n de menor de $\alpha$. ] %Teniendo esta y la anterior, las siguientes funciones son triviales ("$\leq$", ">" y "$\geq$")]
    }

    \paramFuncion{
      \InterfazFuncion{Copiar}{\In{a}{$\alpha$}}{$\alpha$}
      {$res \igobs a$}
      [$\Theta(copy(a))$]
      [funci\'{o}n de copia de $\alpha$]
    }

  \end{paramFormales}

  \seExplicaCon{colaPrior($\alpha$)}

  \generos{heap($\alpha$)}

  \Encabezado{Operaciones del heap}

    \InterfazFuncion{vac\'{i}o}{}{heap($\alpha$)}
    [true]
    {$res \igobs$ vac\'{i}o}
    [$\Theta(1)$]
    [Constructor por defecto de heap($\alpha$)]

    \InterfazFuncion{encolar}{\Inout{h}{heap($\alpha$)}, \In{a}{$\alpha$}}{}{}
    [$h \igobs h_{0}$]
    {$h \igobs$ encolar($a$, $h_{0}$)}
    [$\Theta(log(h.longitud))$]
    [Agrega un elemento a la cola de prioridades]

    \InterfazFuncion{vac\'{i}o?}{\In{h}{heap($\alpha$)}}{bool}
    [true]
    {$h \igobs$ vac\'{i}o()}
    [$\Theta(1)$]
    [Devuelve true si y solo si h es un heap vac\'{i}o]

    \InterfazFuncion{pr\'{o}ximo}{\In{h}{heap($alpha$)}}{$\alpha$}
    [$\not$(vac\'ia?(h))]    % $\land$ $h_{0} \igobs h$]
    {$res \igobs$ pr\'{o}ximo (h)}
    [$\Theta(1)$]
    [Devuelve el pr\'{o}ximo elemento en el heap]

    \InterfazFuncion{desencolar}{\Inout{h}{heap($alpha$)}}{}
    [$\not$(vac\'ia?(h))]
    {$res \igobs$ desencolar(h)}
    [$\Theta(log(h.longitud))$]
    [Elimina el pr\'{o}ximo elemento en el heap]

    \InterfazFuncion{desencolar2}{\Inout{h}{heap($alpha$)}}{$\alpha$}
    [$h \igobs h_{0} \land \not$(vac\'ia?(h))]
    {$res \igobs$ pr\'{o}ximo($h_{0}$) $\land$ $h$ = desencolar($h_{0}$)}
    [$\Theta(log(h.longitud))$]
    [Elimina el pr\'{o}ximo elemento en el heap]

\end{Interfaz}

\begin{Representacion}

  \begin{Estructura}{heap($\alpha$)}[vector($\alpha$)]
  \end{Estructura} 

  \Rep[vector($\alpha$)][v]{
    ($\paratodo{i}{nat}$) ((($2 \* i < $long($v$)) $\impluego$ i\'esimo($2 \* i$, $v$) $\leq$ i\'esimo($i$, $v$)) $\land$ (($2 \* i + 1< $long($v$)) $\impluego$ i\'esimo($2 \* i + 1$, $v$) $\leq$ i\'esimo($i$, $v$)))
  
  %1) si A es el padre de B entonces A >= B
  %2) no hay ningun agujero en el vector es decir estan todos cargados consecutivos. por ende el arbol del heap se carga nivel a nivel de izq a der sin dejar ningun nivel con algun nodo sin hijos
  %3) si A es el padre de B entonces (A = hp[i]) entonces ((B=hp[2i+1]) xor (B=hp2i+2)) 
  %4) (es redundante pero creo que agrega mas claridad)
  %    si A es el padre de B entonces (A=hp[i]) entonces (A=hp[((i/2)-((i+1)%2)))])
  }

  \AbsFc[vector($\alpha$)]{colaPrior($\alpha$)}[v]{
    \IF long($v$) = 0 THEN vac\'{i}a() ELSE encolar(prim($v$), Abs(fin($v$))) FI
  }

\end{Representacion}

\begin{Algoritmos}

  \begin{algoritmo}{iVacio}{}{vector($\alpha$)}
    $res \gets Vac\'ia()$ \;
  \end{algoritmo}
  \datosAlgoritmo{} % Descripción
  {} % Pre
  {} % Post
  {$\Theta(1)$} % Complejidad
  {La funci\'{o}n Vac\'{i}o() de vector es $\Theta(1)$ por ende Vac\'{i}o() de heap es $\Theta(1)$. } % Justificacíón

  \begin{algoritmo}{iEncolar}{\Inout{hp}{heap($\alpha$)}, \In{a}{$\alpha$}}{}
    $hp$.push\_back($a$)\; \com*{$\Theta(1)$}
    \tipo{nat} $i \gets$ $hp$.longitud - 1\; \com*{$\Theta(1)$}
    \tipo{nat} $p \gets$ (i/2) - ((i+1)\%2)\; \com*{$\Theta(1)$}
    \While{$hp$[$p$] < $hp$[$i$]}{ \com*{$\Theta(log(n))$}
      iSwap($hp$, $p$, $i$)\; \com*{$\Theta(copy(\alpha))$}
      $i \gets p$\; \com*{$\Theta(1)$}
      $p \gets$ ($i$/2) - (($i$+1)\%2)\; \com*{$\Theta(1)$}
    }
  \end{algoritmo}
  \datosAlgoritmo{} % Descripción
  {} % Pre
  {} % Post
  {$\Theta(log(n) \times copy(\alpha))$} % Complejidad
  {El while se repite como maximo $log(n)$ veces, donde $n$ es la longitud del vector, y cada iteracion del while tiene complegidad $copy(\alpha)$. } % Justificacíón

  \begin{algoritmo}{iVac\'{i}o?}{\In{v}{vector($\alpha$)}}{bool}
    $res \gets$ esVac\'io?($v$) \; \com*{$\Theta(1)$}
  \end{algoritmo}
  \datosAlgoritmo{} % Descripción
  {} % Pre
  {} % Post
  {$\Theta(1)$} % Complejidad
  {} % Justificacíón

  \begin{algoritmo}{iDesencolar}{\Inout{v}{vector($\alpha$)}}{$\alpha$}
    \tipo{nat} $i \gets$ $v$.longitud - 1\;
    iSwap($v$, $0$, $i$)\; \com*{$\Theta(copy(\alpha))$}
    $res \gets$ $v$.pop\_back()\; \com*{$\Theta(1)$}
    \tipo{nat} $hijo_{0} \gets$ 1\; \com*{$\Theta(1)$}
    \tipo{nat} $hijo_{1} \gets$ 2\; \com*{$\Theta(1)$}
    \tipo{bool} $estaOrdenado \gets$ false\; \com*{$\Theta(1)$}
    \While{$\not$estaOrdenado}{ \com*{$\Theta(log(n))$}
      \eIf{$hijo_{0}$ < $v$.longitud}{
        \eIf{$hijo_{1}$ < $v$.longitud} {
          %//ambos hijos son validos
          \eIf{$v$[$hijo_{0}$] $\geq$ $v$[$hijo_{1}$]}{
            %//hijo0 es mas grande que hijo1
            \eIf{$v$[$hijo_{0}$] > $v$[$i$]}{
              %//hijo0 es mas grande que el padre -> los swapeo
              iSwap($v$, $hijo_{0}$, $i$)\; \com*{$\Theta(copy(\alpha))$}
              $i \gets hijo_{0}$\; \com*{$\Theta(1)$}
            }{
              %//el hijo mas grande es menor a el padre...ESTA ORDENADO!
              $estaOrdenado \gets$ True\; \com*{$\Theta(1)$}
            }
          }{
            %//hijo1 es mas grande que hijo0
            \eIf{$v$[$hijo_{1}$] > $v$[$i$]}{
              %//hijo1 es mas grande que el padre -> los swapeo
              iSwap($v$, $hijo_{1}$, $i$)\; \com*{$\Theta(copy(\alpha))$}
              $i \gets hijo_{1}$\; \com*{$\Theta(1)$}
            }{
              %//el hijo mas grande es menor a el padre...ESTA ORDENADO!
              $estaOrdenado \gets$ True\; \com*{$\Theta(1)$}
            }
          }
        }{
          %//solo hijo0 es valido
          \eIf{$v$[$hijo_{0}$] > $v$[$i$]} {
            %//hijo0 es mas grande que el padre -> los swapeo
            iSwap($v$ ,$hijo_{0}$, $i$)\; \com*{$\Theta(copy(\alpha))$}
            $i \gets hijo_{0}$\; \com*{$\Theta(1)$}
          }{
            %//el hijo mas grande es menor a el padre...ESTA ORDENADO!
            $estaOrdenado \gets$ True\; \com*{$\Theta(1)$}
          }
        }
      }{
        %//ningun hijo es valido por ende no tiene hijos entonces esta ordenado
        $estaOrdenado \gets$ True\; \com*{$\Theta(1)$}
      }
      $hijo_{0} \gets$ 2x$i$+1\; \com*{$\Theta(1)$}
      $hijo_{1} \gets$ 2x$i$+2\; \com*{$\Theta(1)$}
    }
  \end{algoritmo}
  \datosAlgoritmo{} % Descripción
  {} % Pre
  {} % Post
  {$\Theta(log(n) \times copy(\alpha))$} % Complejidad
  {El while se repite como m\'{a}ximo $log(n)$ veces, y cada iteracion del while tiene como complegidad $copy(\alpha)$} % Justificacíón

  \begin{algoritmo}{iSwap}{\Inout{v}{vector($\alpha$)}, \In{a}{$nat$}, \In{b}{$nat$}}{}
    \tipo{$\alpha$} $c$\; \com*{$\Theta(1)$}
    $c \gets v$[$a$]\; \com*{$\Theta(copy(\alpha))$}
    $v$[$a$] $\gets v$[$b$]\; \com*{$\Theta(copy(\alpha))$}
    $v$[$a$] $\gets v$[$c$]\; \com*{$\Theta(copy(\alpha))$}
  \end{algoritmo}
  \datosAlgoritmo{} % Descripción
  {} % Pre
  {} % Post
  {$\Theta(copy(\alpha))$} % Complejidad
  {} % Justificacíón

\end{Algoritmos}


TEXTO DE PRUEBA

\section{M\'{o}dulo DCNet}

\Encabezado{Notas preliminares}
  En todos los casos, al indicar las complejidades de los algoritmos, las variables que se utilizan corresponden a:
  \vspace{-0.5em}\begin{itemize}
    \item $n$: N\'umero de computadoras en la red.
    \item $k$: Longitud de la cola de paquetes m\'as larga al momento.
    \item $L$: Longitud de nombre de computadora m\'as largo de la red.
    \item $i$: Mayor cantidad de interfaces que tiene alguna computadora en la red en el momento.
  \end{itemize}

\servUsados{diccLog, Heap, Red, diccStr, tupla, conjunto }

\begin{Interfaz}
  
  \seExplicaCon{DCNet}
  
  \generos{\tipo{dcnet}}
  
  \Encabezado{Operaciones b\'asicas de lista} % ¿?

  \InterfazFuncion{iniciarDCNet}{\In{r}{Red}}{dcnet}%
  [true]%pre
  {res $\igobs$ iniciarDCNet(r)}%pos
  [$\Theta(n^3 \times I^2 \times L + n^2 \times n!)$]%complejidad
  [Genera un nuevo DCNet sin paquetes. ]%descripcion
  []%aliasing
  
  \InterfazFuncion{crearPaquete}{\Inout{D}{DCNet}, \In{p}{paquete}}{}%
  [D $\igobs$ $D_{0}$ $\land$ $\not$(($\exists p'$: paquete)(paqueteEnTransito?(D,p') $\land$ id(p') = id(p)) $\land$ origen(p) $\in$ computadoras(red(D)) $\yluego$ destino(p) $\in$ computadoras(red(D)) $\yluego$ hayCamino?(red(D), origen(p), destino(p)))]%pre
  {D $\igobs$ crearPaquete($D_{0}$, p)}%pos
  [$\Theta(L+log(k))$]%complejidad
  [Crea un nuevo paquete que no exite en el DCNet anterior. ]%descripcion
  []%aliasing

  \InterfazFuncion{avanzarSegundo}{\Inout{D}{DCNet}}{}%
  [D $\igobs$ $D_{0}$]%pre
  {D $\igobs$ avanzarSegundo($D_{0}$)}%pos
  [$\Theta(n \times log(k))$]%complejidad
  [Avanza un segundo en el DCNet, moviendo todos los paquetes correspondientes. ]%descripcion
  []%aliasing

  \InterfazFuncion{red}{\In{D}{DCNet}}{red}%
  [true]%pre
  {esAlias(res $\igobs$ red(D))}%pos
  [$\Theta(1)$]%complejidad
  [Devuelve la red donde esta funcionando el DCNet. ]%descripcion
  [Pasamos la red por referencia]%aliasing

  \InterfazFuncion{caminoRecorido}{\In{D}{DCNet}, \In{p}{paquete}}{secu(compu)}%
  [paqueteEnTransito?(D, p)]%pre
  {res $\igobs$ caminoRecorrido(D, p)}%pos
  [$\Theta(n \times log(k))$]%complejidad
  [Devuelve la secuencia que contiene de forma ordenada todas las computadoras por las que fue pasando. ]%descripcion
  []%aliasing

  \InterfazFuncion{cantidadEnviados}{\In{D}{DCNet}, \In{c}{compu}}{nat}%
  [c $\in$ computadoras(red(D))]%pre
  {res $\igobs$ cantidadEnviados(D, c)}%pos
  [$\Theta(L)$]%complejidad
  [Devuelve la cantidad de paquetes que envi\'o la computadora ``c''. ]%descripcion
  []%aliasing

  \InterfazFuncion{enEspera}{\In{D}{DCNet}, \In{c}{compu}}{conj(paquete)}%
  [c $\in$ computadoras(red(D))]%pre
  {esAlias(res $\igobs$ enEspera(D, c))}%pos
  [$\Theta(L)$]%complejidad
  [Devuelve los paquetes que tiene en espera la compu ``c''. ]%descripcion
  [El conjunto de paquetes en espera se devuelve por referencia]%aliasing

  \InterfazFuncion{paqueteEnTr\'ansito?}{\In{D}{DCNet}, \In{p}{paquete}}{bool}%
  [true]%pre
  {res $\igobs$ paqueteEnTransito?(D, p)}%pos
  [$\Theta(n \times log(k))$]%complejidad
  [Devuelve ``True'' si y solo si el paquete esta en los paquetes en espera de alguna computadora. ]%descripcion
  []%aliasing

  \InterfazFuncion{laQueM\'asEnvi\'o}{\In{D}{DCNet}}{compu}%
  [true]%pre
  {res $\igobs$ laQueM\'asEnvi\'o(D)}%pos
  [$\Theta(1)$]%complejidad
  [Devuelve una de las computadoras con mas paquetes enviados]%descripcion
  []%aliasing

%  \InterfazFuncion{NOMBRE}{PARAMTROS}{RES}%
%  []%pre
%  {}%pos
%  []%complejidad
%  []%descripcion
%  []%aliasing

\end{Interfaz}
\pagebreak
\begin{Representacion}

  \begin{Estructura}{dcnet}[estrDCNet]

    \begin{Tupla}[estrDCNet]
      \tupItemNL{red}{red}
      \tupItemNL{IDsCompusPorIP}{diccStr(string, nat)}
      \tupItemNL{siguientesCompus}{ad(ad(nat))}
      \tupItemNL{paquetesEnEspera}{ad(tupla(enConjunto: conj(paquete), porID: diccLog(ID, tupla(iPaquete: itConj(paquete),codOrigen: nat, codDestino: nat), porPrioridad: colaPrior(tupla(prioridad, itConjunto(paquete))))))}
      \tupItemNL{\#PaqEnviados}{ad(nat)}
      \tupItemNL{laQueM\'asEnvi\'o}{compu}
      \tupItemNL{IPsCompuPorID}{ad(compu)}
    \end{Tupla}

  \end{Estructura} 

\textbf{Rep en castellano:}

    \begin{enumerate} 
      \item Tam(PaquetesEnEspera) $=$ Tam($\#$PaquetesEnviados) $=$ Tam(ProximaCompu) $=$ Tam(IPsCompusPorID) $=$ $\#$(computadoras(red)). Ademas, la longitud de cada uno de los arreglos de ProximaCompu es igual a \#(computadoras(red)).
      \item La que mas envio, es la computadora correspondiente al maximo del arreglo \#PaquetesEnviados.
      \item Los significados del diccStr IDsCompusPorIP, son consecutivos, iniciando en 0 hasta n.
      \item compus(red) $=$ claves(diccStr)
      \item Para todas las posiciones del arreglo paquetesEnEspera, se debe cumplir lo siquiente:
              \begin{enumerate} 
              \item Los iteradores del AVL tienen siguiente.
              \item Los iteradores del heap tienen siguiente
              \item Todos los iteradores del AVL, apuntan a un paquete que se encuentra en el conjunto.
              \item Todas las claves del AVL, son las IDs de los paquetes apuntados por el iterador que se encuentra en sus significados.
              \item Todos los iteradores del heap, apuntan a un paquete que se encuentra en el conjunto.
              \item Las prioridades de los paquetes apuntados por los iteradores del heap, se corresponden con la prioridad indicada por el heap.
              \item No hay dos iteradores en el heap apuntando al mismo paquete.
              \item La cantidad de paquetes en el conjunto, en el heap y en el AVL es la misma.
              \item En los significados del AVL, el origen y el destino, se corresponden con el origen y el destino del paquete apuntado por el iterador.
              \end{enumerate}
      \item No hay paquetes repetidos entre las distintas posiciones de paquetesEnEspera.
      \item Para cada computadora en el arreglo IPsCompusPorID, usando su IP como clave en el diccStr IDsCompusPorIP, se obtiene como significado la posicion donde se encuentra (en el arreglo).
      \item Para todo (i,j) que esten definidos en siguientesCompus, siguientesCompus[i][j] es la primer compu de alguno de los caminos minimos para llegar desde i a j.
      \item Todas las computadoras del arreglo IPsCompusPorID son las mismas que las computadoras de red.
      \item No hay repetidos en el arreglo IPsCompusPorID


    \end{enumerate}



	\Rep[e][l]{
	($\forall e$: estrDCNet)(
\begin{enumerate} 
%1
\item (tam($e$.paquetesEnEspera) = tam($e$.IPcompusPorID) = tam($e$.siguienteCompu) = tam($e$.\#PaqEnviados) = \#(computadoras($e$.estrRed))) $\land$ 
		($\forall n$: nat)(definido?($e$.siguienteCompu,n) $\impluego$ tam($e$.siguienteCompu[n]) = \#(computadoras($e$.estrRed))) $\land$

%2
\item IPsCompusPorID[PosMaxima($e$.\#PaqEnviados)] = $e$.laQueM\'asEnvi\'o $\land$
%3 
\item ($\forall c$: compu)(c $\in$ computadoras($e$.estrRed) $\impluego$ obtener($\Pi_1$(c),$e$.IDcompusPorIP) < \#(computadoras($e$.estrRed))) $\land$
		(($\forall c_1$, $c_2$: compu)(($c_1 \in$ computadoras($e$.estrRed) $\land$ ($c_2 \in$ computadoras($e$.estrRed)) $\land$ ($c_1 \neq c_2$)) $\impluego$ ((obtener($\Pi_1$($c_1$), $e$.IDcompusPorIP) $\neq$ obtener($\Pi_1$($c_2$), $e$.IDcompusPorIP))))) $\land$

%4
\item(dameNombres(computadoras($e$.estrRed)) = claves(IDcompusPorIP)) $\land$

%5
\item	($\forall L$: nat)( 0 $\leq$ $L$ < tam($e$.paqEnEspera) $\impluego$ (
 \begin{enumerate}

      \item ($\forall$ $it_1$: ItConj(paquete)) $it$ $\in$ dame$\Pi_1$(juntarSignificados($\Pi_2$($e$.paquetesEnEspera[$L$]))) $\impluego$ haySiguiente?($it_1$) $\land$ 

      \item ($\forall$ $it_2$: ItConj(paquete)) $it_2$ $\in$ dame$\Pi_2$(colaAConj($\Pi_3$($e$.paquetesEnEspera[$L$]))) $\impluego$ haySiguiente?($it_2$) $\yluego$ 

			\item ($\forall i$: ItConj(paquete))(siguiente(i) $\in$ (dameSiguientes(dame$\Pi_1$(juntarSignificados$_1$( $\Pi_2$($e$.paquetesEnEspera[$L$])))$\impluego$ siguiente(i) $\in$ $\Pi_1$($e$.paquetesEnEspera[$L$])))) $\land$ 

			\item ($\forall c$: ID) (c $\in$ claves($\Pi_2$($e$.paquetesEnEspera[$L$]))) $\impluego$ $\Pi_1$(siguiente($\Pi_1$(obtener(c, $\Pi_2$($e$.paquetesEnEspera[$L$]))))) = c $\land$ 

      \item ($\forall$ $it$:ItConj(paquete)) siguiente($it$) $\in$ dameSiguientes(dame$\Pi_2$(colaAConj( $\Pi_3$($e$.paquetesEnEspera[$L$])))) $\impluego$ siguiente($it$) $\in$ $\Pi_1$($e$.paquetesEnEspera[$L$]) 

			\item ($\forall$ $t$: tupla(prioridad,ItConj(paquete))) $t$ $\in$ colaAConj($\Pi_3$($e$.paquetesEnEspera[$L$])) $\Rightarrow$ siguiente($\Pi_2$($t$)).prioridad = $\Pi_1$($t$) 

      \item ($\forall$ $it_{1}$: ItConj(paquete)) ($\forall$ $it_{2}$: ItConj(paquete)) $it_{1}$ $\in$ dame$\Pi_{2}$(colaAConj($\Pi_{3}$($e$.paquetesEnEspera[$L$]))) $\land$ $it_{2}$ $\in$ dame$\Pi_{2}$(colaAConj( $\Pi_{3}$($e$.paquetesEnEspera[$L$]))) $it_{1}$ $\neq$ $it_{2}$ $\impluego$ siguiente($it_{1}$) $\neq$ siguiente($it_{2}$) 

      \item \#($\Pi_1$($e$.paquetesEnEspera[L])) = \#claves($\Pi_2$($e$.paquetesEnEspera[L])) $\land$ = \#colaAConj($\Pi_3$($e$.paquetesEnEspera[L])) 

      \item ($\forall c$:ID) $c \in$ claves($\Pi_2$($e$.paquetesEnEspera) $\impluego$ obtener(siguiente($\Pi_1$(obtener($c$, $\Pi_2$($e$.paquetesEnEspera))).origen, $e$.IDsCompusPorIP) = $\Pi_2$(obtener($c$, $\Pi_2$($e$.paquetesEnEspera)))) $\land$ obtener(siguiente($\Pi_1$(obtener($c$, $\Pi_2$($e$.paquetesEnEspera))).destino, $e$.IDsCompusPorIP) = $\Pi_3$(obtener($c$, $\Pi_2$($e$.paquetesEnEspera))))

  \end{enumerate}
%6
\item     ($\forall x$, $z$: nat)((0 $\leq$ x < tam($e$.paquetesEnEspera) $\land$ 0 $\leq$ z < tam($e$.paquetesEnEspera) $\land$ x $\neq$ z) $\impluego$ ($\Pi_1$($e$.paquetesEnEspera[x]) $\cap$ $\Pi_1$($e$.paquetesEnEspera[z])) = $\emptyset$) $\land$ 

%7
\item	($\forall i$: nat)(0 $\leq$ i < \#(computadoras($e$.estrRed)) $\impluego$ obtener($e$.IPcompusPorID[$i$].IP, $e$.IDcompusPorID) = i) $\land$ 

%8
\item	($\forall n$, $m$: nat)(0 $\leq$ n < \#(computadoras($e$.estrRed)) $\land$ 0 $\leq$ m < \#(computadoras($e$.estrRed)) $\impluego$ ($\exists$ $x$: (secu(compu))) $x$ $\in$ caminosminimos($e$.estrRed, $e$.IPcompusPorID[n], $e$.IPcompusPorID[m]) $\land$ prim($x$) $=$ $e$.siguienteCompu[n][m] 

%9
\item    ($\forall i$: nat)(0 $\leq$ i $\leq$ \#computadoras($e$.estrRed) $\impluego$ $e$.IPcompusPorID[$i$] $\in$ computadoras($e$.estrRed)) $\land$ 

%10
\item    ($\forall x$, $y$: nat)((0 $\leq$ x < \#computadoras($e$.estrRed) $\land$ 0 $\leq$ y < \#computadoras($e$.estrRed) $\land$ $x$ $\neq$ y) $\impluego$ $e$.IPcompusPorID[x] $\neq$ $e$.IPcompusPorID[y]) $\land$ 


    \end{enumerate} 
    
    }\mbox{} %habia un parentesis no se de donde sale (estaba cerrando en esta linea)

\pagebreak

\AbsFc[estrDCNet]{DCNet}[e]{red($d$) = $e$.Red $\land$ \\

  ($\forall p$: paquete) paqueteEnTr\'ansito?($d$, $p$) $\impluego$ caminoRecorrido($d$, $p$) = caminoDelPaquete($e$.siguienteCompu, $e$.IPsCompusPorID, $e$.PaquetesEnEspera, $p$, obtener($e$.IDsCompusPorIP, $p$.origen.IP), obtener($e$.IDsCompusPorIP, $p$.destino.IP)) $\land$ \\

  ($\forall c$: compu) $c$ $\in$ computadoras(red($d$)) $\impluego$ cantidadEnviados($d$, $c$) = $e$.\#PaqEnviados[obtener($c$.IP, $e$.IDsCompusPorIP)] $\land$ \\

  ($\forall c$: compu) $c$ $\in$ computadoras(red($d$)) $\impluego$ enEspera($d$, $c$) = enConjunto($e$.paquetesEnEspera[obtener($c$.IP, $e$.IDsCompusPorIP)])
}

  ~

  \textbf{Funciones Auxiliares:}

  ~

\tadOperacion{caminoDelPaquete}{ad(ad(nat)), ad(compu), ad(tupla(enConjunto: conj(paquete){,} porID: diccLog, tupla(iPaquete: itConj(paquete){,}codOrigen: nat{,} codDestino: nat{,} porPrioridad: colaPrior(tupla(prioridad{,} itConjunto(paquete)))))), paquete, nat, nat}{secu(compu)}{$(\exists L$:nat) 0 $\leq$ L < tam($e$.paquetesEnEspera) $\yluego$ def?($p$.ID,  $\Pi_2$($e$.paquetesEnEspera[L])) }   

\small\begin{verbatim}
\caminoDelPaquete: ad(ad(nat)), ad(compu), ad(tupla(enConjunto: conj(paquete), 
porID: diccLog, tupla(iPaquete: itConj(paquete),codOrigen: nat, 
 codDestino: nat, porPrioridad: colaPrior(tupla(prioridad, 
 itConjunto(paquete)))))), paquete, nat, nat -> secu(compu)  
\end{verbatim}

\tadAxioma{caminoDelPaquete($t$, $CsxID$, $ps$, $p$, $compuActual$, $d$)}{\IF def?(ID($p$), porID($ps$[$compuActual$])) THEN $CsxID$[$compuActual$] $\puntito$ <> ELSE $CsxID$[$compuActual$] $\puntito$ caminoDelPaquete($t$, $CsxID$, $ps$, $p$, $t$[$compuActual$][$d$], $d$) FI}

  ~

  \tadOperacion{colaAConj}{colaPrior(tupla<prioridad{,} ItConj(paquete)>)}{conj(tupla<prioridad{,} ItConj(paquete)>)}{}
  \tadAxioma{colaAConj($c$)}{\IF vacia?(c) THEN $\emptyset$ ELSE ag(proximo(c), colaAConj(desencolar(c))) FI}

  ~
  
  \tadOperacion{dame$\Pi_{1}$}{conj(tupla(itConj(paquete), nat, nat))}{conj(ItConj(paquete))}{}
  \tadAxioma{dame$\Pi_{1}$($c$)}{\IF $\emptyset$?(c) THEN $\emptyset$ ELSE ag($\Pi_{1}$(dameuno(c)), dame$\Pi_{1}$(sinuno(c))) FI}

  ~

  \tadOperacion{dame$\Pi_{2}$}{conj(tupla<prioridad, ItConj(paquete)>)}{conj(ItConj(paquete))}{}
  \tadAxioma{dame$\Pi_{2}$($c$)}{\IF $\emptyset$?(c) THEN $\emptyset$ ELSE ag($\Pi_{2}$(dameuno(c)), dame$\Pi_{2}$(sinuno(c))) FI}

  ~
  
  \tadOperacion{dameSiguientes}{conj(ItConj(paquete))}{conj(paquete)}{}
  \tadAxioma{dameSiguientes($c$)}{\IF $\emptyset$?(c) THEN $\emptyset$ ELSE ag(siguiente(dameuno(c)), dameSiguientes(sinuno(c))) FI}

  ~

  \tadOperacion{PosMaxima}{ad(nat)/a}{nat}{tam($a$)>0}
  \tadAxioma{PosMaxima($a$)}{Posicion(Maximo($a$, $a$[0]), $a$)}

  ~

  \tadOperacion{Maximo}{ad(nat), nat}{nat}{}
  \tadAxioma{Maximo($a$, $n$)}{\IF (tam($a$)$=0$) THEN n ELSE {\IF ($a$[0]>$n$) THEN Maximo(finAd(a), a[0]) ELSE Maximo(finAd(a), n) FI} FI}

  ~

  \tadOperacion{Posicion}{nat, ad(nat)}{nat}{definido?($a$,$n$)}
  \tadAxioma{Posicion($n$, $a$)}{Posicion\fAux($a$, $n$, 0)}

  ~

  \tadOperacion{Posicion\fAux}{ad(nat), nat, nat}{nat}{}
  \tadAxioma{Posicion\fAux($a$, $n$, $i$)}{\IF ($a$[0]=$n$) THEN $i$ ELSE Posicion\fAux(finAd($a$), $n$, $i+1$) FI}

  ~

  \tadOperacion{dameNombres}{conj(compu)}{conj(IP)}{}
  \tadAxioma{dameNombres($c$)}{\IF $\emptyset$?($c$) THEN $\emptyset$ ELSE Ag(dameUno($c$.IP, dameNombres(sinUno($c$)))) FI}

  ~

  \tadOperacion{JuntarSignificados}{dicc(IP{,} tupla(itConj(paquete){,} nat{,} nat))}{conj(tupla(itConj(paquete), nat, nat))}{}
  \tadAxioma{JuntarSignificados($d$)}{JuntarSignificados\fAux($d$, claves($d$))}

  ~

  \tadOperacion{JuntarSignificados\fAux}{dicc(IP{,} tupla(itConj(paquete){,} nat{,} nat)), conj(IP)}{conj(tupla(itConj(paquete){,} nat{,} nat))}{}
  \tadAxioma{JuntarSignificados\fAux($d$, $c$)}{\IF $\emptyset$?($c$) THEN $\emptyset$ ELSE Ag(obtener(dameUno($c$), $d$), JuntarSignificados\fAux($d$, sinUno($c$))) FI}

  ~

  \tadOperacion{colaAConj}{colaPrior(tupla(Prioridad{,} itConj(paquete)))}{conj(tupla(prioridad, itConj(paquete)))}{}
  \tadAxioma{colaAConj($c$)}{\IF vacia?($c$) THEN $\emptyset$ ELSE Ag(proximo($c$), colaAConj(desencolar($c$))) FI}

\end{Representacion}

\begin{Algoritmos}

  \SetKwFunction{crearArr}{CrearArreglo}
  \SetKwFunction{crearItRed}{CrearItRed}
  \SetKwFunction{card}{Cardinal}
  \SetKwFunction{compus}{Computadoras}

  \begin{algoritmo}{iIniciarDCNet}{\In{r}{Red}}{DCNet}
    $res$.red $\gets$ \copiar{$r$}\com*{$\Theta(n \times I \times L)$}   
    $res$.\#PaqEnviados $\gets$ \crearArr{cantCompus($res$.red)} \com*{$\Theta(n)$}   
    $res$.IPsCompuPorID $\gets$ \crearArr{cantCompus($res$.red)} \com*{$\Theta(n)$}   
    $res$.siguientesCompus $\gets$ \crearArr{cantCompus($res$.red)} \com*{$\Theta(n)$}   
    $res$.paquetesEnEspera $\gets$ \crearArr{cantCompus($res$.red)} \com*{$\Theta(n)$}   
    \tipo{conj(compu)} $c \gets$ computadoras(r) \com*{$\Theta(1)$}  
    \tipo{itConj} $it_{1} \gets$ crearItConj($c$) \com*{$\Theta(1)$}   
    \tipo{nat} $j \gets$ 0\com*{$\Theta(1)$}   
    \While(\com*[f]{$\Theta(n)$} iteraciones){$j$ < \card{\compus{$res$.red}}}{
      $res$.siguientesCompus[$j$] $\gets$ \crearArr{cantCompus($res$.red)} \com*{$\Theta(n)$}   
      $res$.\#PaqEnviados[$j$] $\gets$ 0\com*{$\Theta(1)$}   
      $res$.paquetesEnEspera[$j$] $\gets$ $\langle$vac\'{i}o(),vac\'{i}o(),vac\'{i}o()$\rangle$\com*{$\Theta(1)$}   
      definir(siguiente($it_{1}$).IP, $j$, $res$.IDsCompusPorIP)\com*{$\Theta(L)$}   
      $res$.IPsCompusPorID[$j$] $\gets$ siguiente($it_{1}$) \com*{$\Theta(1)$}   
      $j \gets j + 1$\com*{$\Theta(1)$}   
      avanzar($it_{1}$)\com*{$\Theta(1)$}   
    }
    \tipo{nat} $k \gets$ 0\com*{$\Theta(1)$}   
    $j \gets 0$\com*{$\Theta(1)$}   
    \While(\com*[f]{$\Theta(n)$} iteraciones){$j$ < Cardinal(Computadoras($res$.red))}{
      \While(\com*[f]{$\Theta(n)$} iteraciones){$k$ < Cardinal(Computadoras($res$.red))}{
        \If(\com*[f]{$\Theta(n \times I)$}){conectadas?($res$.red, $res$.IPsCompusPorID[j], $res$.IPsCompusPorID[k])}{
          \tipo{itConj} $it_{2} \gets$ crearIt(caminosMinimos($res$.red, $res$.IPsCompusPorID[j], $res$.IPsCompusPorID[k]))\;
          $res$.siguientesCompus[$j$][$k$] $\gets$ prim(fin(siguiente($it_{2}$)))\com*{$\Theta(1)$}   
        }
        $k \gets k + 1$\com*{$\Theta(1)$}  
      }
      $j \gets j + 1$\com*{$\Theta(1)$}  
    }
  \end{algoritmo}
 \datosAlgoritmo{} % Descripción
  {} % Pre
  {} % Post
  {$\Theta(n \times I + 4 n + n \times (n + I^2 + L) + n \times (n \times (n + I + n^2 \times n!)))$ = $\Theta(n \times I + n + n^2+ n \times I^2 + n \times L + n^3 + n^2 \times I + n^2 \times n!)$ = $\Theta(n^3 \times I^2 \times L + n^2 \times n!)$} % Complejidad
  {} % Justificacíón

  \begin{algoritmo}{iCrearPaquete}{\Inout{d}{DCNet},\In{p}{paquete}}{}
    % Crear e inicializar variables:
    \tipo{nat} $o \gets$ Obtener($p$.origen, $d$.IDsCompusPorIP)\com*{$\Theta(L)$}
    \tipo{nat} $dest \gets$ Obtener($p$.destino, $d$.IDsCompusPorIP)\com*{$\Theta(L)$}
    $it \gets$ CrearIt(($d$.paquetesEnEspera[$o$]).enConjunto)\com*{$\Theta(1)$}
    $it \gets$ Agregar(($d$.paquetesEnEspeta[$o$]).enConjunto, $p$)\com*{$\Theta(k)$}
    Definir($d$.paquetesEnEspera[$o$].porID, $p$.ID, $\langle$ $it$, $o$, $dest$ $\rangle$)\com*{$\Theta(log(k))$}
    encolar($d$.paquetesEnEspera[$o$].porPrioridad, $it$)\com*{$\Theta(log(k))$}
  \end{algoritmo}
  \datosAlgoritmo{} % Descripción
  {} % Pre
  {} % Post
  {$\Theta(L+log(k))$} % Complejidad
  {} % Justificacíón

  \begin{algoritmo}{iRed}{\In{d}{DCNet}}{Red}
    $res \gets$ $d$.Red \com*{$\Theta(1)$}
  \end{algoritmo}
  \datosAlgoritmo{} % Descripción
  {} % Pre
  {} % Post
  {$\Theta(1)$} % Complejidad
  {} % Justificacíón

  \begin{algoritmo}{iAvanzarSegundo}{\Inout{d}{DCNet}}{}
    \tipo{nat} $j \gets 0$\com*{$\Theta(1)$}
    \tipo{nat} $o$\com*{$\Theta(1)$}
    \tipo{nat} $dest$\com*{$\Theta(1)$}
    \tipo{paquete} $paq$\com*{$\Theta(1)$}
    \tipo{ItConj(paquete)} $i$ \com*{$\Theta(1)$}
    \While(\com*[f]{$\Theta(n)$}){$j$ < Cardinal(Computadoras($d$.red))}{
      \If(\com*[f]{$\Theta(1)$}){!(EsVacio?($d$.paquetesEnEspera[$j$]).enConjunto)}{
        $paq \gets$ Siguiente(desencolar(($d$.paquetesEnEspera[$j$]).porPrioridad))\com*{$\Theta(log(k))$}
        $o \gets$ (Obtener(($d$.paquetesEnEspera[$j$]).porID, $paq$.ID)).codOrigen\com*{$\Theta(log(k))$}
        $dest \gets$ (Obtener(($d$.paquetesEnEspera[$j$]).porID, $paq$.ID)).codDestino\com*{$\Theta(log(k))$}
        $i \gets$ (Obtener(($d$.paquetesEnEspera[$j$]).porID, $paq$.ID)).iPaquete\com*{$\Theta(log(k))$}
        Borrar(($d$.paquetesEnEspera[$j$]).porID, $paq$.ID)\com*{$\Theta(log(k))$}
        EliminarSiguiente(i)\com*{$\Theta(1)$}
        $d$.\#paqEnviados[$j$]++\com*{$\Theta(1)$}
        \If(\com*[f]{$\Theta(1)$}){!($d$.siquienteCompu[$j$][$dest$] = $dest$)}{
          \tipo{itConj(paquete)} $it \gets$ crearIt(($d$.paquetesEnEspera[$d$.siguienteCompu[$j$][$dest$]]).enConjunto)\com*{$\Theta(1)$}
          $it \gets$ AgregarRapido(($d$.paquetesEnEspera[$d$.siguienteCompu[$j$][$dest$]]).enConjunto, $p$)\com*{$\Theta(I)$ lo tomamos como $\Theta(1)$}
          Definir($d$.paquetesEnEspera[$d$.siguienteCompu[$j$][$dest$]].porID, $p$.ID, $\langle$ $it$, $d$.siguienteCompu[$j$][$dest$], $dest$, $\rangle$)\com*{$\Theta(log(k))$}
          encolar($d$.paquetesEnEspera[$d$.siguienteCompu[$j$][$dest$]].porPrioridad, $it$)\com*{$\Theta(log(k))$}
        }
      }
    }
    \tipo{nat} $k \gets 0$\com*{$\Theta(1)$}
    \tipo{nat} $h \gets 0$\com*{$\Theta(1)$}
      \While(\com*[f]{$\Theta(n)$ iteraciones}){k<Cardinal(Computadoras($d$.red))}{
        \If(\com*[f]{$\Theta(1)$}){$d$.\#paqEnviados[$k$]>d.\#paqEnviados[h]}{
          $h \gets k$\com*{$\Theta(1)$}
          k++\com*{$\Theta(1)$}
        }
    }
    $d$.laQueM\'asEnvi\'o $\gets$ $d$.IPsCompusPorID[h]\com*{$\Theta(1)$}
  \end{algoritmo}
  \datosAlgoritmo{} % Descripción
  {} % Pre
  {} % Post
  {$\Theta(n \times log(k))$ esta incluido en $\Theta(n \times (L+log(k)+log(n))$} % Complejidad
  {} % Justificacíón


  \begin{algoritmo}{iCaminoRecorrido}{\In{d}{DCNet}, \In{p}{paquete}}{lista(compu)}
    \tipo{nat} $j \gets 0$\com*{$\Theta(1)$}
    $res \gets$ Vacia()\com*{$\Theta(1)$}
    \While(\com*[f]{$\Theta(log(k) \times n)$}){!(iDefinido?(($d$.paquetesEnEspera[$j$]).porID), $p$.ID)}{
      $j$++\com*{$\Theta(1)$}
    }
    \tipo{nat} $o$\com*{$\Theta(1)$}
    $o \gets$ (Obtener(($d$.paquetesEnEspera[$j$]).porID, $p$.ID)).codOrigen\com*{$\Theta(log(k))$}
    \tipo{nat} $dest$\com*{$\Theta(1)$}
    $dest \gets$ (Obtener(($d$.paquetesEnEspera[$o$]).porID, $p$.ID)).codDestino\com*{$\Theta(log(k)$}
    \While(\com*[f]{$\Theta(n \times log(k))$}){!(Definido?(($d$.paquetesEnEspera[$o$]).porID), $p$.ID)}{
      AgregarAtras($res$, $d$.IPsCompusPorID[o])\com*{$\Theta(I)$ lo tomamos como $\Theta(1)$}
      $o \gets$ $d$.siguientesCompus[o][dest]\com*{$\Theta(1)$}
    }
    AgregarAtras($res$, $d$.IPsCompusPorID[$o$])\com*{$\Theta(I)$ lo tomamos como $\Theta(1)$}
  \end{algoritmo} 
  \datosAlgoritmo{} % Descripción
  {} % Pre
  {} % Post
  {$\Theta(n \times log(k))$} % Complejidad
  {} % Justificacíón

  \begin{algoritmo}{iCantidadEnviados}{\In{d}{DCNet}, \In{c}{compu}}{nat}
    \tipo{nat} $i \gets$ Obtener($d$.IDsCompusPorIP, $c$.IP)\com*{$\Theta(L)$}
    $res \gets$ $d$.\#paqEnviados[$i$]\com*{$\Theta(1)$}
  \end{algoritmo}
 \datosAlgoritmo{} % Descripción
  {} % Pre
  {} % Post
  {$\Theta(L)$} % Complejidad
  {} % Justificacíón

  \begin{algoritmo}{iEnEspera}{\In{d}{DCNet}, \In{c}{compu}}{conj(Paquete)}
    \tipo{nat} $i \gets$ Obtener($d$.IDsCompusPorIP, $c$.IP)\com*{$\Theta(L)$}
    $res \gets$ ($d$.paquetesEnEspera[$i$]).enConjunto\com*{$\Theta(1)$}
  \end{algoritmo}
 \datosAlgoritmo{} % Descripción
  {} % Pre
  {} % Post
  {$\Theta(L)$} % Complejidad
  {} % Justificacíón

  \begin{algoritmo}{iLaQueM\'{a}sEnvio}{\In{d}{DCNet}}{compu}
    $res \gets$ $d$.laQueM\'{a}sEnvio \com*{$\Theta(1)$}
  \end{algoritmo}
  \datosAlgoritmo{} % Descripción
  {} % Pre
  {} % Post
  {$\Theta(1)$} % Complejidad
  {Se pasa la computadora por referencia} % Justificacíón

  \begin{algoritmo}{paqueteEntr\'{a}nsito?}{\In{d}{DCNet}, \In{p}{paquete}}{bool}
    $res \gets$ false \com*{$\Theta(1)$}
    \tipo{nat} $i \gets 0$\com*{$\Theta(1)$}
    \While(\com*[f]{$\Theta(n)$}){$i$<longitud($d$.paquetesEnEspera)}{
      \If(\com*[f]{$\Theta(log(k))$}){definido?($p$, ($d$.paquetesEnEspera[$i$]).porID)}{
          $res \gets$ true\com*{$\Theta(1)$}
      }
    }
  \end{algoritmo}
  \datosAlgoritmo{} % Descripción
  {} % Pre
  {} % Post
  {$\Theta(n \times log(k))$} % Complejidad
  {} % Justificacíón

\end{Algoritmos}



\end{document}