\documentclass[10pt, a4paper]{article}
\usepackage[paper=a4paper, left=1.5cm, right=1.5cm, bottom=1.5cm, top=3.5cm]{geometry}
\usepackage[latin1]{inputenc}
\usepackage[T1]{fontenc}
\usepackage[spanish,activeacute]{babel}
\usepackage{indentfirst}
\usepackage{fancyhdr}
\usepackage{latexsym}
\usepackage{lastpage}
\usepackage{aed2-symb,aed2-itef,aed2-tad,caratula}
\usepackage[colorlinks=true, linkcolor=blue]{hyperref}
\usepackage{calc}

\newcommand{\f}[1]{\text{#1}}
\newcommand{\fAux}{$_{\text{aux}}$}
\renewcommand{\paratodo}[2]{\ensuremath{\forall~#2: \text{#1}}}

% Macros de diseño provistas por la cátedra %

\usepackage{xspace}
\usepackage{xargs}
\usepackage{ifthen}

\newcommand{\moduloNombre}[1]{\textbf{#1}}

\let\NombreFuncion=\textsc
\let\TipoVariable=\texttt
\let\tipo=\texttt
\let\ModificadorArgumento=\textbf
\newcommand{\res}{$res$\xspace}
\newcommand{\tab}{\hspace*{7mm}}

\newcommandx{\TipoFuncion}[3]{%
  \NombreFuncion{#1}(#2) \ifx#3\empty\else $\to$ \res\,: \TipoVariable{#3}\fi%
}
\newcommand{\In}[2]{\ModificadorArgumento{in} \ensuremath{#1}\,: \TipoVariable{#2}\xspace}
\newcommand{\Out}[2]{\ModificadorArgumento{out} \ensuremath{#1}\,: \TipoVariable{#2}\xspace}
\newcommand{\Inout}[2]{\ModificadorArgumento{in/out} \ensuremath{#1}\,: \TipoVariable{#2}\xspace}
\newcommand{\Aplicar}[2]{\NombreFuncion{#1}(#2)}

\newlength{\IntFuncionLengthA}
\newlength{\IntFuncionLengthB}
\newlength{\IntFuncionLengthC}
%InterfazFuncion(nombre, argumentos, valor retorno, precondicion, postcondicion, complejidad, descripcion, aliasing)
\newcommandx{\InterfazFuncion}[9][4=true,6,7,8,9]{%
  \hangindent=\parindent
  \TipoFuncion{#1}{#2}{#3}\\%
  \textbf{Pre} $\equiv$ \{#4\}\\%
  \textbf{Post} $\equiv$ \{#5\}%
  \ifx#6\empty\else\\\textbf{Complejidad:} #6\fi%
  \ifx#7\empty\else\\\textbf{Descripci\'on:} #7\fi%
  \ifx#8\empty\else\\\textbf{Aliasing:} #8\fi%
  \ifx#9\empty\else\\\textbf{Requiere:} #9\fi%
}

\newenvironment{Interfaz}{%
  \parskip=2ex%
  \noindent\textbf{\Large Interfaz}%
  \par%
}{}

\newenvironment{Representacion}{%
  \vspace*{2ex}%
  \noindent\textbf{\Large Representaci\'on}%
  \vspace*{2ex}%
}{}

\newenvironment{Algoritmos}{%
  \vspace*{2ex}%
  \noindent\textbf{\Large Algoritmos}%
  \vspace*{2ex}%
}{}


\newcommand{\Encabezado}[1]{
  \vspace*{1ex}\par\noindent\textbf{\large #1}\par
}

\newenvironmentx{Estructura}[2][2={estr}]{%
  \par\vspace*{2ex}%
  \TipoVariable{#1} \textbf{se representa con} \TipoVariable{#2}%
  \par\vspace*{1ex}%
}{%
  \par\vspace*{2ex}%
}%

\newboolean{EstructuraHayItems}
\newlength{\lenTupla}
\newenvironmentx{Tupla}[1][1={estr}]{%
    \settowidth{\lenTupla}{\hspace*{3mm}donde \TipoVariable{#1} es \TipoVariable{tupla}$($}%
    \addtolength{\lenTupla}{\parindent}%
    \hspace*{3mm}donde \TipoVariable{#1} es \TipoVariable{tupla}$($%
    \begin{minipage}[t]{\linewidth-\lenTupla}%
    \setboolean{EstructuraHayItems}{false}%
}{%
    $)$%
    \end{minipage}
}

\newcommandx{\tupItem}[3][1={\ }]{%
    %\hspace*{3mm}%
    \ifthenelse{\boolean{EstructuraHayItems}}{%
        ,#1%
    }{}%
    \emph{#2}: \TipoVariable{#3}%
    \setboolean{EstructuraHayItems}{true}%
}

\newcommandx{\tupItemNL}[3][1={\ }]{%
    %\hspace*{3mm}%
    \ifthenelse{\boolean{EstructuraHayItems}}{%
        ,\\#1%
    }{}%
    \emph{#2}: \TipoVariable{#3}%
    \setboolean{EstructuraHayItems}{true}%
}

\newcommandx{\RepFc}[3][1={estr},2={e}]{%
  \tadOperacion{Rep}{#1}{bool}{}%
  \tadAxioma{Rep($#2$)}{#3}%
}%

\newcommandx{\Rep}[3][1={estr},2={e}]{%
  \tadOperacion{Rep}{#1}{bool}{}%
  \tadAxioma{Rep($#2$)}{true \ssi #3}%
}%

\newcommandx{\Abs}[5][1={estr},3={e}]{%
  \tadOperacion{Abs}{#1/#3}{#2}{Rep($#3$)}%
  \settominwidth{\hangindent}{Abs($#3$) \igobs #4: #2 $\mid$ }%
  \addtolength{\hangindent}{\parindent}%
  Abs($#3$) \igobs #4: #2 $\mid$ #5%
}%

\newcommandx{\AbsFc}[4][1={estr},3={e}]{%
  \tadOperacion{Abs}{#1/#3}{#2}{Rep($#3$)}%
  \tadAxioma{Abs($#3$)}{#4}%
}%

\newcommand{\DRef}{\ensuremath{\rightarrow}}

% Macros de diseño propias %

\usepackage{scrextend} % Para poder indentar bloques

\newenvironment{paramFormales}{
  \textbf{par\'ametros formales}
  \vspace{-0.5em}
  \list{}{\leftmargin8em \topsep0.2em \itemsep0.25em \labelsep2em}
}{
  \endlist 
}

\newcommand{\paramGeneros}[1]{\item[\textbf{g\'eneros}] #1}

\newcommand{\paramFuncion}[1]{\item[\textbf{funci\'on}] \parbox[t]{\textwidth-2\parindent-1.7cm}{#1}}

\newcommand{\seExplicaCon}[1]{\parbox{3cm}{\textbf{se explica con}:} \tadNombre{#1}}

\newcommand{\generos}[1]{\parbox{3cm}{\textbf{g\'eneros}:} #1}

\usepackage[noresetcount]{algorithm2e}
\usepackage{float}

\NoCaptionOfAlgo\SetAlgoLongEnd\LinesNumbered\SetAlgoLined\RestyleAlgo{ruled}\IncMargin{1em}\DontPrintSemicolon\SetArgSty{}\SetCommentSty{textsf}\SetFuncSty{textsf}

\newenvironment{algoritmo}[3]{
  \setcounter{AlgoLine}{0}
  \begin{algorithm}[H]
  \caption{\TipoFuncion{#1}{#2}{#3}}
}{
  \end{algorithm}
  \vspace{0em}
}

\newenvironment{contAlgoritmo}[1]{
  \begin{algorithm}[H]
  \caption{\NombreFuncion{#1} \emph{(cont.)}}
}{
  \end{algorithm}
}

\newcommand{\datosAlgoritmo}[5]{
  \ifx#1\empty\else \textbf{Descripci\'on:} #1

  \fi \ifx#2\empty\else\textbf{Pre} $\equiv$ \{#2\}

  \fi \ifx#3\empty\else\textbf{Post} $\equiv$ \{#3\}

  \fi \textbf{Complejidad:} #4 

  \ifx#5\empty\else\textbf{Justificaci\'on:} #5 \fi \vspace{1em}
}

\SetKwComment{com}{ $\triangleright$ }{}
\SetKwFunction{copiar}{Copiar}
\SetKwFunction{delete}{delete}
\def\NULL{\textrm{NULL}}
\def\new{\textbf{\&}}

\sloppy

\hypersetup{%
 % Para que el PDF se abra a pagina completa.
 pdfstartview= {FitH \hypercalcbp{\paperheight-\topmargin-1in-\headheight}},
 pdfauthor={Grupo 11},
 pdfkeywords={trabajo, pr\'actico, algoritmos, dise\~no},
 pdfsubject={Trabajo pr\'actico 2 - Dise\~no - DCNet}
}

\parskip=5pt % 10pt es el tamaño de fuente

% Pongo en 0 la distancia extra entre ítemes.
\let\olditemize\itemize
\def\itemize{\olditemize\itemsep=0pt}

% Acomodo fancyhdr.
\pagestyle{fancy}
\thispagestyle{fancy}
\addtolength{\headheight}{1pt}
\lhead{Algoritmos y Estructuras de Datos II}
\rhead{Trabajo Pr\'actico 2 - Dise\~no - DCNet}
\cfoot{\thepage /\pageref{LastPage}}
\renewcommand{\footrulewidth}{0.4pt}

\author{Grupo 11}
\date{}
\title{Trabajo pr\'actico 2 - Dise\~no - DCNet}

\def\Materia{Algoritmos y Estructuras de Datos II}
\def\Titulo{{\small De los creadores de sacarCompu...} \\ \vspace*{1.5ex} Trabajo pr\'actico 2}
\def\Subtitulo{Dise\~no - DCNet}
\def\Grupo{Grupo 11}
\integrante{Frizzo, Franco}{013/14}{francofrizzo@gmail.com}
\integrante{Mart\'inez, Manuela}{160/14}{martinez.manuela.22@gmail.com}
\integrante{Rabinowicz, Luc\'ia}{105/14}{lu.rabinowicz@gmail.com}
\integrante{Weber, Andr\'es}{923/13}{herr.andyweber@gmail.com}

\begin{document}

\maketitle
\newpage\null\thispagestyle{empty}\newpage

\section{Peque\~nos m\'odulos}

\subsection{M\'{o}dulo IP}
  \servUsados{string}
  \Encabezado{Representaci\'on}
    \begin{Estructura}{IP}[string]
    \end{Estructura} 

\subsection{M\'{o}dulo Interfaz}
  \servUsados{nat}
  \Encabezado{Representaci\'on}
    \begin{Estructura}{Interfaz}[nat]
    \end{Estructura} 

\subsection{M\'{o}dulo Prioridad}
  \servUsados{nat}
  \Encabezado{Representaci\'on}
    \begin{Estructura}{Prioridad}[nat]
    \end{Estructura} 

\subsection{M\'{o}dulo ID}
  \servUsados{nat}
  \Encabezado{Representaci\'on}
    \begin{Estructura}{ID}[nat]
    \end{Estructura} 

\subsection{M\'{o}dulo Paquete}
  \servUsados{tupla, ID, prioridad, compu}
  \Encabezado{Interfaz}
    \InterfazFuncion{$\bullet$ < $\bullet$}{\In{a}{paquete}, \In{b}{paquete}}{bool}%
    [true]%pre
    {$res$ $\igobs$ $a$.prioridad < $b$.prioridad}%pos
    [$\Theta(1)$]%complejidad
    [Da una relacio\'on de orden a los paquetes por prioridad. ]%descripcion
    []%aliasing
  \Encabezado{Representaci\'on}
    \begin{Estructura}{Paquete}[paquete]
      \begin{Tupla}[paquete]
        \tupItem{ID}{ID}
        \tupItem{prioridad}{prioridad}
        \tupItem{origen}{compu}
        \tupItem{destino}{compu}
      \end{Tupla}
    \end{Estructura}
  \begin{algoritmo}{i<}{\In{a}{paquete}, \In{a}{paquete}}{bool}
    $res$ $\gets$ ($a$.prioridad < $b$.prioridad)\com*{$\Theta(1)$}
  \end{algoritmo}
  \datosAlgoritmo{} % Descripción
  {} % Pre
  {} % Post
  {$\Theta(1)$} % Complejidad
  {} % Justificacíón  

\subsection{M\'{o}dulo Compu}
  \servUsados{tupla, IP, conjunto, interfaz}
  \Encabezado{Representaci\'on}
    \begin{Estructura}{Compu}[compu]
      \begin{Tupla}[compu]
        \tupItem{IP}{IP}
        \tupItem{Interfaces}{conj(interfaz)}
      \end{Tupla}
    \end{Estructura}


\section{M\'{o}dulo Red}

\begin{Interfaz}
  
  \seExplicaCon{Red, Iterador Bidireccional(Compu)}

  \generos{\tipo{red}, \tipo{itRed}}
  
  \Titulo{Operaciones b\'{a}sicas de red}

  \InterfazFuncion{iniciarRed}{}{Red}%
  {res $\igobs$ iniciarRed()}%pos
  [$\Theta(1234)$]%complejidad
  [Genera una nueva red sin ninguna computadora.]%descripcion
  []%aliasing
  
  \InterfazFuncion{agregarCompu}{\Inout{R}{Red}, \In{c}{compu}}{}%
  [$R$ $\igobs$ $R_{0}$ $\land$ ($\forall c'$: compu)($c' \in$ computadoras($r$) $\rightarrow$ ip($c$) $\neq$ ip($c'$))]%pre
  {$R$ $\igobs$ agregarCompu($R_{0}$, c)}%pos
  [$\Theta(1324)$]%complejidad
  [Agrega una nueva pc a una red.]%descripcion
  []%aliasing

  \InterfazFuncion{conectar}{\Inout{r}{Red}, \In{c_0}{compu}, \In{i_0}{interfaz}, \In{c_1}{compu}, \In{i_1}{interfaz}}{red}%
  [$R$ $\igobs$ $R_{0}$ $\land c_{1} \in $computadoras($r$) $\land c_{2} \in$ computadoras($r$) $\land$ ip($c_{0} $) $\neq$ ip($c_{1} $) $\land \not$conectadas?($r$, $c_{0}$ ,$c_{1} $) $\land \not$ usaInterfaz?($r$, $c_{0}$, $i_{0}$) $\land \not$ usaInterfaz?($r$, $c_{1}$, $i_{1}$)]%pre
  {$R$ $\igobs$ conectar($R_{0}$, $c_{0}$, $i_{0}$, $c_{1}$, $i_{1}$)}%pos
  [$\Theta(1234)$]%complejidad
  [Conecta la pc $c_{0}$ con la pc $c_{1}$ a trav\'es de las interfaces $i_{0}$ y $i_{1}$ respectivamente. ]%descripcion
  []%aliasing
  
  \InterfazFuncion{computadoras}{\In{R}{Red}}{conj(compus)}%
  {$res$ $\igobs$ computadoras(R) } %es alias(res,compus(R))             %pos
  [$\Theta(1234)$]%complejidad
  [Devuelve todas las computadoras de la red. ]%descripcion
  []%aliasing
  
  \InterfazFuncion{conectadas?}{\In{r}{Red}, \In{c_{0}}{compu}, \In{c_{1}}{compu}}{bool}%
  [$c_{0}$ $\in$ computadoras($R$) $\land c_{1} \in $ computadoras($R$)]%pre
  {$res$ $\igobs$ conectadas?($R$, $c_{0}$, $c_{1}$)}%pos
  [$\Theta(1234)$]%complejidad
  [Devuelve true si y solo si la pc $c_{0}$ esta conectada a la pc $c_{1}$]%descripcion
  []%aliasing
  
  \InterfazFuncion{interfazUsada}{\In{R}{Red}, \In{c_{0}}{compu}, \In{c_{1}}{compu}}{interfaz}%
  [$c_{0}$ $\in$ computadoras($R$) $\land c_{1} \in $ computadoras($R$) $\yluego$ conectadas?($R$, $c_{0}$, $c_{1}$)]%pre
  {$res$ $\igobs$ interfazUsada($R$, $c_{0}$, $c_{1}$)}%pos
  [$\Theta(1234)$]%complejidad
  [Devuelve la interfaz usada por $c_{0}$ para conectarse a $c_{1}$]%descripcion
  []%aliasing
  
  \InterfazFuncion{vecinos}{\In{R}{Red}, \In{c}{compu}}{conj(compus)}%
  [$c$ $\in$ computadoras($R$)]%pre
  {$res$ $\igobs$ vecinos($R$, $c$)}%pos
  [$\Theta(1234)$]%complejidad
  [Devuelve el conjunto de vecinos de la pc $c$]%descripcion
  []%aliasing
  
  \InterfazFuncion{usaInterfaz?}{\In{R}{Red}, \In{c}{compu}, \In{i}{interfaz}{bool}}%
  [$c$ $\in$ computadoras($R$)]%pre
  {$res$ $\igobs$ usaInterfaz?($R$, $c$, $i$)}%pos
  [$\Theta(1234)$]%complejidad
  [Devuelve true si y solo si la pc $c$ esta usando la interfaz $i$. ]%descripcion
  []%aliasing
  
  \InterfazFuncion{caminosMinimos}{\In{R}{Red}, \In{c_{0}}{compu}, \In{c_{1}}{compu}}{conj(secu(compus))}%
  [$c_{0}$ $\in$ computadoras($R$) $\land c_{1} \in $ computadoras($R$)]%pre
  {$res$ $\igobs$ caminosMinimos($R$, $c_{0}$, $c_{1}$)}%pos
  [$\Theta(1234)$]%complejidad
  [Devuelve todos los caminos m\'nimos posibles entre $c_{0}$ y $c_{1}$. De no haber ninguno, devuelve $\emptyset$. ]%descripcion
  []%aliasing
  
  \InterfazFuncion{hayCamino?}{\In{R}{Red}, \In{c_{0}}{compu}, \In{c_{1}}{compu}}{bool}%
  [$c_{0}$ $\in$ computadoras($R$) $\land c_{1} \in $ computadoras($R$)]%pre
  {$res$ $\igobs$ hayCamino?($R$, $c_{0}$, $c_{1}$)}%pos
  [$\Theta(1234)$]%complejidad
  [Devuelve true si y solo si hay alg\'un camino posible entre $c_{0}$ y $c_{1}$. ]%descripcion
  []%aliasing

\end{Interfaz}

\begin{Representacion}

  \begin{Estructura}{red}[estrRed]

    \begin{Tupla}[estrRed]
      \tupItem{compus}{lista(estrCompu)}
      \tupItem{cantidadCompus}{nat}
    \end{Tupla}

    \begin{Tupla}[estrCompu]
      \tupItem{IP}{string}
      \tupItem{conexiones}{lista(tupla(interfaz, itLista(estrCompu)))}
    \end{Tupla}

  \end{Estructura}

  \begin{Estructura}{itRed}[itLista(estrCompu)]
  \end{Estructura}

  \RepFc[Red][e]{($\forall c$: compu)(c $\in$ ArmarComputadoras(e.conexiones) $\impluego$ $\lnot$ Pertenece?(e.conexiones, c, c))$\land$ \\
    $\#$ArmarComputadoras(e.conexiones) = e.cantidadCompus $\land$ \\
    ($\forall c_{1}$: compu)(($\forall c_{2}$: compu) ($c_{1}$ $\in$ ArmarComputadoras(e.conexiones) $\land$ $c_{2}$ $\in$ ArmarComputadoras(e.conexiones) $\impluego$ Pertenece?(e.conexiones, $c_{1}$, $c_{2}$) $\Leftrightarrow$ Pertenece?(e.conexiones, $c_{2}$, $c_{1}$))) $\land$ \\
    ($\forall c_{1}$: compu)($c_{1}$ $\in$ ArmarComputadoras(e.conexiones) $\impluego$ ($\forall c_{2}$: compu) (Pertenece?(e.conexiones, $c_{1}$, $c_{2}$) $\Rightarrow$ $c_{2}$ $\in$ ArmarComputadoras(e.conexiones))) $\land$ \\
    sinRepetidos(ArmarSecuencia(e.conexiones)) 
    %FALTAN SINREPETIDOS Y ARMAR SECUENCIA
    }\mbox{}

  ~

  \AbsFc[estrRed]{Red}[e]{(r: Red | computadoras(r) = ArmarComputadoras(e.conexiones) $\land$ \\
    ($\forall c_{1}$: compu)(($\forall c_{2}$: compu) conectados?(r, $c_{1}$, $c_{2}$) = Pertenece?(e.conexiones, $c_{1}$, $c_{2}$) $\land$ \\
    InterfazUsada(r, $c_{1}$, $c_{2}$) = DevolverInterfaz(e.conexiones, $c_{1}$, $c_{2}$)))
  }

  ~      

  \tadOperacion{ArmarComputadoras}{lista(tupla<string, lista(tupla<Interfaz, ItRed>)>)}{conj(compu)}{} %falta arreglar comas
  \tadAxioma{ArmarComputadoras($l$)}{\IF vacia?(l) THEN $\emptyset$ ELSE Ag(<$\pi_{1}$(prim($l$)), GenerarInterfaces($\pi_{2}$(prim($l$)))>, ArmarComputadoras(fin($l$))) FI}

  ~

  \tadOperacion{GenerarInterfaces}{lista(tupla<Interfaz, ItLista(estrCompu)>)}{conj(Interfaz)}{}
  \tadAxioma{GenerarInterfaces($l$)}{\IF vacia?($l$) THEN $\emptyset$ ELSE Ag($\pi_{1}$(prim($l$)), GenerarInterfaces(fin($l$))) FI}

  ~

  \tadOperacion{Pertenece?}{lista(tupla<string, lista(tupla<Interfaz, ItRed>)>)/l, compu/c_{1}, compu/c_{2}}{bool}{}
  \tadAxioma{Pertenece?($l$, $c_{1}$, $c_{2}$)}{\IF ($\pi_{1}$(prim($l$) = $\pi_{1}$($c_{1}$))) THEN $\pi_{1}$($c_{2}$) $\in$ GenerarCompus($\pi_{2}$(prim($l$))) ELSE Pertenece?(fin($l$), $c_{1}$, $c_{2}$) FI}

  ~

  \tadOperacion{GenerarCompus}{lista(tupla<Interfaz, ItLista(estrCompu)>)}{conj(string)}{}
  \tadAxioma{GenerarCompus($l$)}{\IF vacia?($l$) THEN $\emptyset$ ELSE Ag($\pi_{1}$(siguiente($\pi_{2}$(prim($l$)))), GenerarCompus(fin($l$))) FI}

  ~

  \tadOperacion{DevolverInterfaz}{lista(tupla<string, lista(tupla<Interfaz, ItRed>)>)/l, compu/c_{1}, compu/c_{2}}{Interfaz}{} % Restriccion? Pertenece?(l, c1, c2) o c1 \in computadoras(r)
  \tadAxioma{DevolverInterfaz($l$, $c_{1}$, $c_{2}$)}{\IF ($\pi_{1}$(prim($l$)) = $\pi_{1}$($c_{1}$)) THEN DevolverInterfazAux($\pi_{2}$(prim($l$), $c_{2}$)) ELSE DevolverInterfaz(fin($l$, $c_{1}$, $c_{2}$)) FI}

  ~  
  
   \tadOperacion{DevolverInterfazAux}{lista(tupla<Interfaz, ItRed>)/l, compu/c}{Interfaz}{} % Restriccion? Pertenece?(l, c1, c2) o c1 \in computadoras(r)
  \tadAxioma{DevolverInterfaz($l$, $c$)}{\IF ($\pi_{1}$($c_{2}$) = $\pi_{1}$(siguiente($\pi_{2}$(prim($l$))))) THEN $\pi_{1}$(prim($l$)) ELSE DevolverInterfazAux(fin($l$, c)) FI}

  ~

\end{Representacion}


\section{M\'odulo DCNet}

\begin{Interfaz}
  
  \textbf{se explica con:} \tadNombre{DCNet}
  
  \textbf{g\'eneros:} \TipoVariable{dcnet}
  
\end{Interfaz}

\begin{Representacion}

  \begin{Estructura}{dcnet}[estrDCNet]

    \begin{Tupla}[estrDCNet]
      \tupItem{red}{red}
      \tupItem{IDsCompusPorIP}{dicc\_trie(string, nat)}
      \tupItem{siguientesCompus}{arregloDimensionable(arregloDimensionable(nat))}
      \tupItem{paqEnEspera}{arregloDimensionable(tupla(conjunto(paquete), colaPrior(tupla(nat, itConjunto(paquete))), dicc\_AVL(itConjunto(paquete))))}
      \tupItem{\#PaqEnviados}{arregloDimensionable(nat)}
      \tupItem{laQueM\'asEnvi\'o}{itRed}
      \tupItem{\'ultimoId}{nat}
    \end{Tupla}

  \end{Estructura} 

  \Rep[e][l]{($\forall e$: estr)(
  (tam(e.paquetesEnEspera) = tam(e.IPcompusXID) = tam(e.siguienteCompu) = tam (e.cantPaquetesEnviados) = #(computaodras(e.estrRed)) $\land$ 
  ($\forall n$: nat)(definido?(e.siguienteCompu,n) $\impluego$ tam(e.siguienteCompu[n]) = #(computadoras(e.estrRed))) $\land$
  Maximo(e.cantPaqEnviados) = e.cantidadEnviados[obtener($\pi_1$(siguiente(e.laQueMasEnvio)),e.IDcompusPorIP)] $\land$
  ($\forall c$: compu)(c $\in$ computadoras(e.estRed) $\impluego$ obtener($\pi_1$(c),e.IDcompusPorIP) $\min$ #(computadoras(e.estrRed))) $\land$ (($\forall c_1$, $c_2$: compu)( ($c_1 \in$ computadoras(e.estRed) $\land$ ($c_2 \in$ computadoras(e.estrRed)) $\land$ ($c_1 \neq c_2$) ) $\impluego$ (( obtener($\pi_1$($c_1$), e.IDcompusPorIP) $\neq$ obtener($\pi_1$($c_2$), e.IDcompusPorIP))))) $\land$
  (dameNombres(computadoras(e.estrRed)) = claus(IDcompusPorIP)) $\land$
  ($\forall L$: nat)( 0 $\leq$ L $\min$ tam(e.paqEnEspera) $\impluego$ 
  %5)
  (
  ($\forall i$: ItConj(paquete))(
  siguiente(i) $\in$ (dameSiguientes(dame$\pi_1$($juntarSignificados_1$($\pi_2$(e.paquetesEnEspera[L])))$\impluego$ siguiente(i) $\in \pi_1$(e.paquetesEnEspera[L]) $\land$ ($\forall c$: ID)(c$\in$ claves($\pi_2$(e.paquetesEnEspera[L])) $\impluego \pi_1$(siguiente($\pi_1$(obtener($\pi_2$(c), e.paquetesEnEspera[L])))) = c ($\forall i$: ItConj(paquete))(siguiente(i)$\in$ dameSiguientes(dame$\pi_1$($juntarSignificados_2$($\pi_3$(e.paquetesEnEspera[L])))) $\impluego$ siguiente(i) $\in \pi_1$(e.paquetesEnEspera[L])  ))))


  )

  )


  )}\mbox{}

  ~      

  \tadOperacion{Nodo}{lst/l,nat}{puntero(nodo)}{$l$.primero $\neq$ NULL}
  \tadAxioma{Nodo($l$,$i$)}{\IF $i = 0$ THEN $l$.primero ELSE Nodo(FinLst($l$), $i-1$) FI}

  ~
\end{Representacion}

\newcommand{\avlKS}{diccAVL(\ensuremath{\kappa}, \ensuremath{\sigma})} % Macro para diccAVL(k, s)

\section{M\'odulo Diccionario AVL($\kappa$, $\sigma$)}

\begin{Interfaz}
  
  \begin{paramFormales}
    \paramGeneros{$\kappa$, $\sigma$}

    \paramFuncion{
      \InterfazFuncion{$\bullet = \bullet$}{\In{k_1}{$\kappa$}, \In{k_2}{$\kappa$}}{\tipo{bool}}
      {$res \igobs (k_1 = k_2)$}
      [$\Theta(equal(k_1, k_2))$]
      [funci\'on de igualdad de $\kappa$]
    }

    \paramFuncion{
      \InterfazFuncion{$\bullet = \bullet$}{\In{k_1}{$\kappa$}, \In{k_2}{$\kappa$}}{\tipo{bool}}
      {$res \igobs (k_1 \leq k_2)$}
      [$\Theta(order(k_1, k_2))$]
      [funci\'on de comparaci\'on por orden total estricto de $\kappa$]
    }

    \paramFuncion{
      \InterfazFuncion{Copiar}{\In{k}{$\kappa$}}{$\kappa$}
      {$res \igobs k$}
      [$\Theta(copy(k))$]
      [funci\'on de copia de $\kappa$]
    }

    \paramFuncion{
      \InterfazFuncion{Copiar}{\In{s}{$\sigma$}}{$\sigma$}
      {$res \igobs s$}
      [$\Theta(copy(s))$]
      [funci\'on de copia de $\sigma$]
    }
  \end{paramFormales}

  \seExplicaCon{Diccionario($\kappa$, $\sigma$)}

  \generos{\tipo{\avlKS}}

  \Encabezado{Operaciones de diccionario}

    \InterfazFuncion{Vacio}{}{\avlKS}
    {$res \igobs$ vac\'io}
    []
    []

    \InterfazFuncion{Definir}{\Inout{d}{\avlKS}, \In{k}{$\kappa$}, \In{s}{$\sigma$}}{}
    [$d \igobs d_0$]
    {$d \igobs$ definir($k$, $s$, $d_0$)}
    []
    []

    \InterfazFuncion{Borrar}{\Inout{d}{\avlKS}, \In{k}{$\kappa$}}{}
    [$d \igobs d_0 \land$ def?($k$, $d$)]
    {$d \igobs$ borrar($k$, $d_0$)}
    []
    []

    \InterfazFuncion{\#Claves}{\In{d}{\avlKS}}{nat}
    {$res \igobs$ \#(claves($d$))}
    []
    []

    \InterfazFuncion{Definido?}{\In{d}{\avlKS}, \In{k}{$\kappa$}}{bool}
    {$res \igobs$ def?($k$, $d$)}
    []
    []

    \InterfazFuncion{Obtener}{\In{d}{\avlKS}, \In{k}{$\kappa$}}{$\sigma$}
    [def?($k$, $d$)]
    {$res \igobs$ obtener($k$, $d$)}
    []
    []

\end{Interfaz}

\begin{Representacion}

  \begin{Estructura}{diccAVL($\alpha$)}[estrAVL]

    \begin{Tupla}[estrAVL]
      \tupItemNL{raiz}{puntero(nodo)}%
      \tupItemNL{cantNodos}{nat}%
    \end{Tupla}

    \begin{Tupla}[nodo]
      \tupItemNL{clave}{$\kappa$}%
      \tupItemNL{significado}{$\sigma$}%
      \tupItemNL{padre}{puntero(nodo)}%
      \tupItemNL{izq}{puntero(nodo)}%
      \tupItemNL{der}{puntero(nodo)}%
      \tupItemNL{altSubarbol}{nat}%
    \end{Tupla}

  \end{Estructura} 

  \Rep[estrAB][ab]{
    ($ab$.cantNodos = 0) = ($ab$.raiz = NULL) $\yluego$ \\
    ($ab$.cantNodos > 0) $\impluego$ ($ab$.cantNodos = cantHijos(\*($ab$.raiz)) + 1) $\land$ \\
    (\paratodo{nodo}{n_1, n_2})(($n_1$ $\in$ nodos($ab$) $\land$ $n_2$ $\in$ nodos($ab$)) $\impluego$ (($n_1$.izq $\neq$ $n_2$.der) $\land$ (($n_1$.izq = $n_2$.izq $\lor$ $n_1$.der = $n_2$.der) $\implies$ ($n_1 = n_2$))))
  }

  \AbsFc[estrAB]{ab($\alpha$)}[ab]{
    \IF $ab$.cantNodos = 0 THEN nil ELSE bin(Abs(plantar($ab$.raiz $\to$ izq)), $ab$.raiz $\to$ dato, Abs(plantar($ab$.raiz $\to$ der))) FI
  }

  \tadOperacion{hijos}{nodo}{conj(nodo)}{}
  \tadAxioma{hijos($n$)}{\IF $n$.izq = NULL THEN $\emptyset$ ELSE Ag(\*($n$.izq), hijos(\*($n$.izq))) FI $\cup$ \IF $n$.der = NULL THEN $\emptyset$ ELSE Ag(\*($n$.der), hijos(\*($n$.der))) FI}

  \tadOperacion{cantHijos}{nodo}{nat}{}
  \tadAxioma{cantHijos($n$)}{\#(hijos($n$))}

  \tadOperacion{nodos}{estrAB}{conj(nodo)}{}
  \tadAxioma{nodos($ab$)}{\IF $ab$.raiz = NULL THEN $\emptyset$ ELSE Ag(\*($ab$.raiz), hijos(\*($ab$.raiz))) FI}

  \tadOperacion{plantar}{puntero(nodo)}{estrAB}{}
  \tadAxioma{plantar($p$)}{$\langle p$, \IF $p$ = NULL THEN 0 ELSE cantHijos(\*($p$)) + 1 FI$\rangle$}

\end{Representacion}

\begin{Algoritmos}
  
  % Definición de funciones que se usan en el algoritmo:
  \SetKwFunction{atajoFuncion}{nombreFuncion}

  \begin{algoritmo}{nombreAlgoritmo}{parametros}{tipoSalida}
    % Crear e inicializar variables:
    \tipo{tipoVariable} $nombreVariable \gets valor$ \;
    % Nota: las líneas sin comentario deben terminar en \;

    % Comentarios:
    Contenido de la linea \tcp*[h]{Comentario pegado al texto, onda C++} \com*{Comentario contra el margen}
    Contenido de la linea \com*[h]{Comentario pegado al texto} \tcp*{Comentario contra el margen, onda C++}

    % Llamada a función definida antes:
    \atajoFuncion{parametros}

    % If/then:
    \If{guarda}{
      Codigo \;
    }

    % If/then/else:
    \eIf(\com*[f]{Comentario opcional}){guarda}{
      Codigo (then) \;
    }(\com*[f]{Otro comentario opcional}){
      Codigo (else) \;
    }

    % If/then/elseif:
    \uIf{guarda}{
      Codigo (then) \;
    }\ElseIf{otra guarda}{
      Codigo (else) \;
    }

    % While:
    \While(\com*[f]{Comentario opcional}){guarda}{
      Codigo \;
    }

  \end{algoritmo}

  \SetKwFunction{fdb}{FDB}
  \SetKwFunction{raizq}{RotarAIzquierda}
  \SetKwFunction{rader}{RotarADerecha}
  \SetKwFunction{buscar}{Buscar}
  \SetKwFunction{max}{max}
  \SetKwFunction{rebalancear}{RebalancearArbol}

  \begin{algoritmo}{iVacio}{}{estrAVL}{
    $res \gets \langle$NULL, 0$\rangle$ \com*{$\Theta(1)$}
  }
  \end{algoritmo}

  \begin{algoritmo}{iRebalancearArbol}{\In{n}{puntero(nodo)}}{}
    \tipo{puntero(nodo)} $p \gets n$ \com*{$\Theta(1)$}
    \While{$p \neq \NULL$}{ 
      \tipo{int} $fdb1 \gets$ \fdb{$p$}\;
      \uIf{$fdb1 = 2$}{
        \tipo{puntero(nodo)} $q \gets (p \to der)$ \;
        \tipo{int} $fdb2 \gets$ \fdb{$q$} \;
        \uIf{$fdb2 = 1 \lor fdb2 = 0$}{
          \raizq{$p$} \;
          $p \gets q$ \;
        }\ElseIf{$fbd2 = -1$}{
          \rader{$q$} \;
          \raizq{$p$} \;
          $p \gets (q \to padre)$ \;
        }
      }\ElseIf{$fdb1 = -2$}{
        \tipo{puntero(nodo)} $q \gets (p \to izq)$ \;
        \tipo{int} $fdb2 \gets$ \fdb{$q$} \;
        \uIf{$fdb2 = -1 \lor fdb2 = 0$}{
          \rader{$p$} \;
          $p \gets q$ \;
        }\ElseIf{$fbd2 = 1$}{
          \raizq{$q$} \;
          \rader{$p$} \;
          $p \gets (q \to padre)$ \;
        }
      }
      $p \gets (p \to padre)$ \;
    }
  \end{algoritmo}

  \begin{algoritmo}{iFBD}{\In{n}{puntero(nodo)}}{int}
    \tipo{int} $altIzq \gets n \to izq = \NULL$ ? 0 : $n \to izq \to altSubarbol$ \;
    \tipo{int} $altDer \gets n \to der = \NULL$ ? 0 : $n \to izq \to altSubarbol$ \;
    $res \gets altDer - altIzq$ \;
  \end{algoritmo}

  \begin{algoritmo}{iRotarAIzquierda}{\In{n}{puntero(nodo)}}{}
    \If{$n.padre \neq \NULL$}{
      \eIf{$n.padre \to izq = n$}{
        $(n.padre \to izq) \gets n.der$ \;
      }{
        $(n.padre \to der) \gets n.der$ \;
      }
    }
    $(n.der \to padre) \gets n.padre$ \; 
    $n.padre \gets n.der$ \;
    $n.der \gets (n.der \to izq)$ \;
    \If{$n.der \neq \NULL$}{
      $(n.der \to padre) \gets n$ \;
    }
    $(n.padre \to izq) \gets n$ \;
    \While{$actual \neq \NULL$}{
      $(actual \to altSubarbol) \gets 1$ + \max{$actual \to izq \to altSubarbol$, $actual \to der \to altSubarbol$} \;
      $actual \gets (actual \to padre)$ \;
    }
  \end{algoritmo}

  \begin{algoritmo}{iRotarADerecha}{\In{n}{puntero(nodo)}}{}
    \If{$n.padre \neq \NULL$}{
      \eIf{$n.padre \to izq = n$}{
        $(n.padre \to izq) \gets n.izq$ \;
      }{
        $(n.padre \to der) \gets n.izq$ \;
      }
    }
    $(n.izq \to padre) \gets n.padre$ \; 
    $n.padre \gets n.izq$ \;
    $n.izq \gets (n.izq \to der)$ \;
    \If{$n.izq \neq \NULL$}{
      $(n.izq \to padre) \gets n$ \;
    }
    $(n.padre \to der) \gets n$ \;
    \While{$actual \neq \NULL$}{
      $(actual \to altSubarbol) \gets 1$ + \max{$actual \to izq \to altSubarbol$, $actual \to der \to altSubarbol$} \;
      $actual \gets (actual \to padre)$ \;
    }
  \end{algoritmo}

  \begin{algoritmo}{iBuscar}{\In{e}{estrAVL}, \In{k}{$\kappa$}, \Out{padre}{puntero(nodo)}}{puntero(nodo)}
    $padre \gets \NULL$ \;
    $actual \gets e.raiz$ \;
    \While{$actual \neq \NULL \yluego (actual \to clave \neq k)$}{ % ¡Preguntar por el y luego!
      $padre \gets actual$ \;
      \eIf{$k \leq (padre \to clave)$}{
        $actual \gets (actual \to izq)$ \;
      }{
        $actual \gets (actual \to der)$ \;
      }
    }
    $res \gets actual$ \;
    \end{algoritmo}

    \begin{algoritmo}{iBorrar}{\Inout{e}{estrAVL}, \In{k}{$\kappa$}}{}
      \tipo{puntero(nodo)} $padre \gets \NULL$ \;
      \tipo{puntero(nodo)} $lugar \gets$ \buscar{$e$, $k$, $padre$} \;
      \uIf{$lugar \to izq = \NULL \land lugar \to der = \NULL$}{
        \eIf{$padre \neq \NULL$}{
          \eIf{$padre \to izq = lugar$}{
            $(padre \to izq) \gets \NULL$ \; 
            $(padre \to altSubarbol) \gets 1 + (padre \to der \to altSubarbol)$ \;
          }{
            $(padre \to der) \gets \NULL$ \; 
            $(padre \to altSubarbol) \gets 1 + (padre \to izq \to altSubarbol)$ \;
          }
          \rebalancear{$padre$} \;
        }{
          $e.raiz = \NULL$ \;
        }
        \delete{$lugar$} \;
      }\uElseIf{$lugar \to der = \NULL$}{
        $(lugar \to izq \to padre) \gets padre$ \;
        \eIf{$padre \neq \NULL$}{
          \eIf{$padre \to izq = lugar$}{
            $(padre \to izq) \gets (lugar \to izq)$ \; 
          }{
            $(padre \to der) \gets (lugar \to izq)$ \; 
          }
          $(padre \to altSubarbol) \gets 1$ + \max{$padre \to der \to altSubarbol$, $padre \to izq \to altSubarbol$} \;
          \rebalancear{$padre$} \;
        }{
          $e.raiz = lugar \to izq$ \;
        }
        \delete{$lugar$} \;
      }\uElseIf{$lugar \to izq = \NULL$}{
        $(lugar \to der \to padre) \gets padre$ \;
        \eIf{$padre \neq \NULL$}{
          \eIf{$padre \to izq = lugar$}{
            $(padre \to izq) \gets (lugar \to der)$ \; 
          }{
            $(padre \to der) \gets (lugar \to der)$ \; 
          }
          $(padre \to altSubarbol) \gets 1$ + \max{$padre \to der \to altSubarbol$, $padre \to izq \to altSubarbol$} \;
          \rebalancear{$padre$} \;
        }{
          $e.raiz = lugar \to izq$ \;
        }
        \delete{$lugar$} \;
      }\Else{
        \com*{Ac\'a viene la parte horrible del algoritmo}
      }
    \end{algoritmo}

\end{Algoritmos}

\section{M\'{o}dulo Heap($\alpha$)}

\begin{Interfaz}
  
  \begin{paramFormales}
    \paramGeneros{$\alpha$}

    \paramFuncion{
      \InterfazFuncion{$\bullet \leq \bullet$}{\In{a_{1}}{$\alpha$}, \In{a_{a}}{$\alpha$}}{bool}
      {$res \igobs (a_{1} \leq a_{2})$}
      [$\Theta(compare(a_{1}, a_{2}))$]
      [funci\'on de comparaci\'on por orden total estricto de $\alpha$]
    }

    \paramFuncion{
      \InterfazFuncion{Copiar}{\In{a}{$\alpha$}}{$\alpha$}
      {$res \igobs a$}
      [$\Theta(copy(a))$]
      [funci\'{o}n de copia de $\alpha$]
    }

  \end{paramFormales}

  \seExplicaCon{Cola de Prioridad($\alpha$)}

  \generos{\tipo{heap($\alpha$)}}

  \Encabezado{Operaciones de cola de prioridad}

    \InterfazFuncion{vac\'{i}o}{}{heap($\alpha$)}
    [true]
    {$res \igobs$ vac\'{i}o}
    [$\Theta(1)$]
    [Devuelve un heap vac\'io.]

    \InterfazFuncion{encolar}{\Inout{h}{heap($\alpha$)}, \In{a}{$\alpha$}}{}{}
    [$h \igobs h_{0}$]
    {$h \igobs$ encolar($a$, $h_{0}$)}
    [$\Theta(log(n))$]
    [Agrega un elemento al heap.]

    \InterfazFuncion{vac\'{i}o?}{\In{h}{heap($\alpha$)}}{bool}
    [true]
    {$h \igobs$ vac\'{i}o()}
    [$\Theta(1)$]
    [Devuelve true si y solo si el heap no tiene elementos.]

    \InterfazFuncion{desencolar}{\Inout{h}{heap($alpha$)}}{}
    [$\not$(vac\'ia?(h))]
    {$res \igobs$ desencolar(h)}
    [$\Theta(log(n))$]
    [Devuelve uno de los elementos de m\'axima prioridad del heap, y lo elimina de la estructura.]

\end{Interfaz}

\begin{Representacion}

  \begin{Estructura}{heap($\alpha$)}[vector($\alpha$)]
  \end{Estructura} 

  \Rep[vector($\alpha$)][v]{
    ($\paratodo{i}{nat}$) ((($2 \* i + 1 < $long($v$)) $\impluego$ i\'esimo($2 \* i + 1$, $v$) $\leq$ i\'esimo($i$, $v$)) $\land$ (($2 \* i + 2 < $long($v$)) $\impluego$ i\'esimo($2 \* i + 2$, $v$) $\leq$ i\'esimo($i$, $v$)))
  }

  \tadOperacion{i\'esimo}{nat/$i$,secu($\alpha$/$s$)}{$\alpha$}{$n \leq long($\alpha$)$}  

  \tadAxioma{i\'esimo($i$, $s$)}{\IF $i = 0$ THEN prim($s$) ELSE i\'esimo($i - 1$, fin($s$)) FI}

  \AbsFc[vector($\alpha$)]{colaPrior($\alpha$)}[v]{
    \IF long($v$) = 0 THEN vac\'{i}a() ELSE encolar(prim($v$), Abs(fin($v$))) FI
  }

\end{Representacion}

\begin{Algoritmos}

  \begin{algoritmo}{iVac\'io}{}{vector($\alpha$)}
    $res \gets Vac\'ia()$ \; \com*{$\Theta(1)$}
  \end{algoritmo}
  \datosAlgoritmo{} % Descripción
  {} % Pre
  {} % Post
  {$\Theta(1)$} % Complejidad
  {} % Justificación

  \begin{algoritmo}{iEncolar}{\Inout{v}{vector($\alpha$)}, \In{a}{$\alpha$}}{}
    agregarAtras($v$, $a$) \; \com*{$\Theta(1)$}
    \tipo{nat} $i \gets $long$(v)$ \; \com*{$\Theta(1)$}
    \While{$i \neq 0 \land v[i] < v[i \text{ div } 2 - (i + 1)\%2]$}{
      swap($v$, $i$, $i \text{ div } 2 - (i + 1)\%2$) \; \com*{$\Theta(copy(\alpha))$}
      $i \gets i \text{ div } 2 - (i + 1)\%2$ \; \com*{$\Theta(1)$}
    }
  \end{algoritmo}
  \datosAlgoritmo{} % Descripción
  {} % Pre
  {} % Post
  {$\Theta(log(n)) \times \Theta(copy(\alpha))$} % Complejidad
  {El while se repite como m\'aximo $log(n)$ veces y cada repetici\'on tiene $\Theta(copy(\alpha))$. } % Justificación

  \begin{algoritmo}{iVac\'{i}o?}{\In{v}{vector($\alpha$)}}{bool}
    $res \gets$ esVac\'io?($v$) \; \com*{$\Theta(1)$}
  \end{algoritmo}
  \datosAlgoritmo{} % Descripción
  {} % Pre
  {} % Post
  {$\Theta(1)$} % Complejidad
  {} % Justificación

  \begin{algoritmo}{iDesencolar}{\Inout{v}{vector($\alpha$)}}{$\alpha$}
    $res \gets v[0]$ \; \com*{$\Theta(1)$}
    \tipo{nat} $i \gets 0$ \; \com*{$\Theta(1)$}
    \While{(($2i + 1 \leq$ long($v$)) $\land$ $v[i] < v[2i + 1]$) $\lor$ ($2i + 2 \leq$ long($v$)) $\land$ $v[i] < v[2i + 2]$)}{
      \EIf{$2i + 2 \leq$ long($v$)}{
        \eIf{$v[i] < v[2i + 2]$}{
          swap($v$, $i$, $2i + 2$) \; \com*{$\Theta(copy(\alpha))$}
          $i \gets 2i + 2$ \; \com*{$\Theta(1)$}
        }{
          swap($v$, $i$, $2i + 1$) \; \com*{$\Theta(copy(\alpha))$}
          $i \gets 2i + 1$ \; \com*{$\Theta(1)$}
        }
      }{
        swap($v$, $i$, $2i + 1$) \; \com*{$\Theta(copy(\alpha))$}
        $i \gets 2i + 1$ \; \com*{$\Theta(1)$}
      }
    }
  \end{algoritmo}
  \datosAlgoritmo{} % Descripción
  {$\not$(vac\'ia?(h))} % Pre
  {} % Post
  {$\Theta(log(n)) \times \Theta(copy(\alpha))$} % Complejidad
  {El while se repite como m\'aximo $log(n)$ veces y cada repetici\'on tiene $\Theta(copy(\alpha))$. } % Justificación

  \begin{algoritmo}{iSwap}{\Inout{v}{vector($\alpha$), \In{i}{nat}, \In{j}{nat}}}
    $\alpha aux \gets v[i]$ \; \com*{$\Theta(copy(\alpha))$}
    $v[i] \gets v[j]$ \; \com*{$\Theta(copy(\alpha))$}
    $v[j] \gets aux$ \; \com*{$\Theta(copy(\alpha))$}
  \end{algoritmo}
  \datosAlgoritmo{} % Descripción
  {} % Pre
  {} % Post
  {$\Theta(copy(\alpha))$} % Complejidad
  {} % Justificación

\end{Algoritmos}


\section{M\'{o}dulo Diccionario Trie}

\begin{Interfaz}
  
  \begin{paramFormales}
    \paramGeneros{$\alpha$}

    \paramFuncion{
      \InterfazFuncion{$\bullet = \bullet$}{\In{a_{0}}{$\alpha$}, \In{a_{1}}{$\alpha$}}{bool}
      {$res \igobs$ equal($a_{1}$, $a_{2}$)}
      [$\Theta(equal(a_{1}, a_{2}))$]
      [funci\'on de igualdad de $\alpha$]
    }
    
    \paramFuncion{
      \InterfazFuncion{Copiar}{\In{a}{$\alpha$}}{$\alpha$}
      {$res \igobs a$}
      [$\Theta(copy(a))$]
      [funci\'{o}n de copia de $\alpha$]
    }

  \end{paramFormales}

  \seExplicaCon{dicc(secu(char), $\alpha$)}

  \generos{diccTrie($\alpha$)}

  \Encabezado{Operaciones de diccionario}

    \InterfazFuncion{vac\'{i}o}{}{diccTrie($\alpha$)}
    [true]
    {$res \igobs$ vac\'{i}o}
    [$\Theta(1)$]
    [Constructor por defecto de diccTrie($\alpha$)]

    \InterfazFuncion{definir}{\Inout{d}{diccTrie($\alpha$)}, \In{k}{string}, \In{s}{$\alpha$}}{diccTrie($\alpha$)}
    [$d \igobs d_{0}$]
    {$d \igobs$ definir($k$, $s$, $d_{0}$}
    [$\Theta(L)$]
    [Define una plabra en el diccTrie($\alpha$)]

    \InterfazFuncion{borrar}{\Inout{d}{diccTrie($\alpha$)}, \In{k}{string}}{diccTrie($\alpha$)}
    [$d \igobs d_{0}$ $\land$ def?($d$, $k$)]
    {$d \igobs$ borrar($k$, $s$, $d_{0}$}
    [$\Theta(L)$]
    [Borra una definicion en el diccTrie($\alpha$)]

    \InterfazFuncion{def?}{\In{d}{diccTrie($\alpha$)}, \In{k}{string}}{bool}
    [true]
    {$res \igobs$ def?($d$, $k$)}
    [$\Theta(L)$]
    [Pregunta si la palabra $k$ esta definida en el diccTrie($\alpha$)]

    \InterfazFuncion{obtener}{\In{d}{diccTrie($\alpha$)}, \In{k}{string}}{$\alpha$}
    [def?($d$, $k$)]
    {$res \igobs$ obtener($d$, $k$)}
    [$\Theta(L)$]
    [Devuelve el significado de la palabra $k$]

\end{Interfaz}


\end{document}